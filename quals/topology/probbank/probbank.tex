\documentclass[12pt]{article}
 
\usepackage[margin=.5in]{geometry} 
\usepackage{amsmath,amsthm,amssymb,outlines}
\usepackage{graphicx,tikzsymbols,tcolorbox}
\renewcommand\qedsymbol{$\blacksquare$}
\newcommand{\Z}{\mathbb{Z}}
\newcommand{\C}{\mathbb{C}}
\newcommand{\R}{\mathbb{R}}
\tcbuselibrary{most}

\newtcolorbox{newtitle}{
  enhanced,
  colframe=black,
  colback=white,
  boxsep=5pt,
  arc=8pt,
  borderline={0.5pt}{0pt}{black},
  borderline={1.8pt}{-5pt}{black},
  after skip=30pt
}

\newtcolorbox[auto counter]{hatcher}[1][]{
  enhanced,
  breakable,
  title={Hatcher \ifx\\#1\\\thetcbcounter\else#1\fi},
  colframe=black,
  colback=white,
  colbacktitle=white,
  fonttitle=\bfseries,
  coltitle=black,
  attach boxed title to top left={yshift=-0.25mm-\tcboxedtitleheight/2,yshifttext=2mm-\tcboxedtitleheight/2, xshift=2mm},
  boxed title style={boxrule=0.5mm}
}

\newtcolorbox[auto counter]{statement}[1][]{
  enhanced,
  breakable,
  title={Exercise \ifx\\#1\\\thetcbcounter\else#1\fi},
  colframe=black,
  colback=white,
  colbacktitle=white,
  fonttitle=\bfseries,
  coltitle=black,
  attach boxed title to top left={yshift=-0.25mm-\tcboxedtitleheight/2,yshifttext=2mm-\tcboxedtitleheight/2, xshift=2mm},
  boxed title style={boxrule=0.5mm}
}

\newtcolorbox{newproof}{
  enhanced,
  breakable,
  frame hidden,
  colback=white,
  title={Proof.},
  fonttitle=\bfseries,
  coltitle=black,
  colbacktitle=white,
  boxed title style={boxrule=0.5mm},
  attach boxed title to top left={yshift=-0.25mm-\tcboxedtitleheight/2,yshifttext=2mm-\tcboxedtitleheight/2, xshift=2mm},
  borderline west={1.5pt}{8pt}{black},
  after upper={\hfill $\blacksquare$}
}

\begin{document}

\begin{newtitle}
  \begin{center}
    \textbf{\Huge Alegbraic Topology Problem Bank}
  \end{center}
  \textbf{Dahlen Elstran} \hfill \textbf{Spring 2025}
\end{newtitle}

\begin{center} \section*{Homework 1} \end{center}

\begin{hatcher}[0.2]
    Construct an explicit deformation retraction of $\mathbb{R}^n - \{0\}$ onto $S^{n-1}$.
\end{hatcher}

\begin{hatcher}[0.3]
    \begin{itemize}
        \item[(a)] Show that the composition of homotopy equivalences $X \to Y$ and $Y \to Z$ is a homotopy equivalence $X \to Z$. Deduce that homotopy equivalence is an equivalence relation.

        \item[(b)] Show that the relation of homotopy among maps $X \to Y$ is an equivalence relation.

        \item[(c)] Show that a map homotopic to a homotopy equivalence is a homotopy equivalence.
    \end{itemize}
\end{hatcher}

\begin{hatcher}[0.6]
  \begin{itemize}
    \item[(a)] Let $X$ be the subspace of $\R^2$ consisting of the horizontal segment $[0,1] \times \{0\}$ 
      together with all the vertical segments $\{r\} \times [0,1-r]$ for $r$ a rational number in 
      $[0,1]$. Show that $X$ deformation retracts to any point in the segment $[0,1] \times \{ 0\}$, but
      not to any other point. [See the preceding problem]
    \item[(b)] Let $Y$ be the subspace of $\R^2$ that is the union of an infinite number of copies of $X$ 
      arranged as in the figure below. Show that $Y$ is contractible but does not deformation retract 
      onto any point. 
    \item[(c)] Let $Z$ be the zigzag subspace of $Y$ homeomorphic to $/R$ indicated by the heavier line. 
      Show there is a deformation retraction in the weak sense (see Exercise 4) of $Y$ onto $Z$, but no true 
      deformation retraction.
  \end{itemize}
\end{hatcher}

\begin{hatcher}[0.10]
    Show that a space $X$ is contractible if and only if every map $f:X \to Y$, for arbitrary $Y$, is nullhomotopic. Similarly, show $X$ is contractible if and only if every map $f: Y \to X$ is nullhomotopic.
\end{hatcher}

\begin{hatcher}[0.11]
    Show that $f: X \to Y$ is a homotopy equivalence if there exist maps $g,h:Y \to X$ such that $fg \cong \text{id}$ and $hf \cong \text{id}$. More generally, show that $f$ is a homotopy equivalence if $fg$ and $hf$ are homotopy equivalences. 
\end{hatcher}

\begin{hatcher}[0.16]
  Show that $S^{\infty}$ is contractible.
\end{hatcher}

\begin{hatcher}[0.17]
    Construct a 2-dimensional cellcomplex that contains both an annulus $S^1 \times I$ and a Mobius band as 
    deformation retractions.
\end{hatcher}

\begin{hatcher}[0.20]
    Show that the subspace $X \subset \mathbb{R}^3$ formed by a Klein bottle intersecting itself in a circle is homotopy equivalent to $S^1 \vee S^1 \vee S^1$.
\end{hatcher}

% \begin{center} \section*{Homework 2} \end{center}

% \begin{center} \section*{Homework 3} \end{center}

% \begin{center} \section*{Homework 4} \end{center}

% \begin{center} \section*{Homework 5} \end{center}

% \begin{center} \section*{Midterm} \end{center}

% \begin{center} \section*{Homework 6} \end{center}

% \begin{center} \section*{Homework 7} \end{center}

% \begin{center} \section*{Homework 8} \end{center}

% \begin{center} \section*{Homework 9} \end{center}

% \begin{center} \section*{Homework 10} \end{center}

% \begin{center} \section*{Homework 11} \end{center}

% \begin{center} \section*{Final} \end{center}

% \begin{center} \section*{Bonus Assignment} \end{center}

\end{document}
