\documentclass[12pt]{article}
 \usepackage[margin=1in]{geometry} 
\usepackage{amsmath,amsthm,amssymb,outlines}
\usepackage{graphicx,tikzsymbols,tcolorbox}
\usepackage{tikz,caption,subcaption}
\newcommand{\C}{\mathbb{C}}
\newcommand{\R}{\mathbb{R}}
\newcommand{\Z}{\mathbb{Z}}
\renewcommand\qedsymbol{$\blacksquare$}
\tcbuselibrary{most}

% Defining new environments

\newtcolorbox{newtitle}{
  enhanced,
  colframe=black,
  colback=white,
  boxsep=5pt,
  arc=8pt,
  borderline={0.5pt}{0pt}{black},
  borderline={1.8pt}{-5pt}{black},
  after skip=30pt
}

\newtcolorbox[auto counter, number within=section]{theorem}[2][title]{
  enhanced,
  title={Theorem \thetcbcounter \if\relax\detokenize{#2}\relax\else\ (#2)\fi},
  colframe=black,
  colback=white,
  colbacktitle=white,
  fonttitle=\bfseries,
  coltitle=black,
  attach boxed title to top left={yshift=-0.25mm-\tcboxedtitleheight/2,yshifttext=2mm-\tcboxedtitleheight/2, xshift=2mm},
  boxed title style={boxrule=0.5mm}
}

\newtcolorbox[]{definition}{
  enhanced,
  title={Definition \thetcbcounter},
  colframe=black,
  colback=white,
  colbacktitle=white,
  fonttitle=\bfseries,
  coltitle=black,
  attach boxed title to top left={yshift=-0.25mm-\tcboxedtitleheight/2,yshifttext=2mm-\tcboxedtitleheight/2, xshift=2mm},
  boxed title style={boxrule=0.5mm}
}

\newtcolorbox[auto counter, number within=section]{fact}{
  enhanced,
  title={Fact \thetcbcounter},
  colframe=black,
  colback=white,
  colbacktitle=white,
  fonttitle=\bfseries,
  coltitle=black,
  attach boxed title to top left={yshift=-0.25mm-\tcboxedtitleheight/2,yshifttext=2mm-\tcboxedtitleheight/2, xshift=2mm},
  boxed title style={boxrule=0.5mm}
}

\newtcolorbox[auto counter, number within=section]{notation}{
  enhanced,
  title={Notation},
  colframe=black,
  colback=white,
  colbacktitle=white,
  fonttitle=\bfseries,
  coltitle=black,
  attach boxed title to top left={yshift=-0.25mm-\tcboxedtitleheight/2,yshifttext=2mm-\tcboxedtitleheight/2, xshift=2mm},
  boxed title style={boxrule=0.5mm}
}

\newtcolorbox{newproof}{
  enhanced,
  breakable,
  frame hidden,
  colback=white,
  title={Proof.},
  fonttitle=\bfseries,
  coltitle=black,
  colbacktitle=white,
  boxed title style={boxrule=0.5mm},
  attach boxed title to top left={yshift=-0.25mm-\tcboxedtitleheight/2,yshifttext=2mm-\tcboxedtitleheight/2, xshift=2mm},
  borderline west={1.5pt}{8pt}{black},
  after upper={\hfill $\blacksquare$}
}

\newtcolorbox[auto counter, number within=section]{uq}{
  enhanced,
  title=Question,
  colframe=red,
  colback=white,
  colbacktitle=white,
  fonttitle=\bfseries,
  coltitle=black,
  attach boxed title to top left={yshift=-0.25mm-\tcboxedtitleheight/2,yshifttext=2mm-\tcboxedtitleheight/2, xshift=2mm},
  boxed title style={boxrule=0.5mm}
}

\newtcolorbox[auto counter, number within=section]{aq}{
  enhanced,
  title=Question,
  colframe=black!50!black,
  colback=white,
  colbacktitle=white,
  fonttitle=\bfseries,
  coltitle=black,
  attach boxed title to top left={yshift=-0.25mm-\tcboxedtitleheight/2,yshifttext=2mm-\tcboxedtitleheight/2, xshift=2mm},
  boxed title style={boxrule=0.5mm}
}

\newtcolorbox{answer}{
  enhanced,
  frame hidden,
  colback=white,
  colframe=black!50!black,
  title={Answer},
  fonttitle=\bfseries,
  coltitle=black,
  colbacktitle=white,
  boxed title style={boxrule=0.5mm},
  attach boxed title to top left={yshift=-0.25mm-\tcboxedtitleheight/2,yshifttext=2mm-\tcboxedtitleheight/2, xshift=2mm},
  borderline west={1.5pt}{8pt}{black!50!black}
}

% Redefine section as well

% Renew commands later

\begin{document}

\begin{newtitle}
  \begin{center}
    \textbf{\Huge Algebraic Topology Qualifying Exam Cheat Sheet}
  \end{center}
  \textbf{Dahlen Elstran} \hfill \textbf{Spring 2025}
\end{newtitle}

\begin{center} \section*{Point Set Topology Definitions} \end{center}

\begin{definition}
  If $X$ is a topological space with topology $\mathcal{T}$, we say that a subset $U$ of $X$ 
  is an \textbf{open set} of $X$ if $U$ belongs to the collection $\mathcal{T}$.
\end{definition}

\begin{definition}
  A subset $A$ of a topological space $X$ is said to be \textbf{closed} if the set $x - A$ is open.
\end{definition}

\begin{definition}
  The \textbf{closure} of $A$ is defined as the intersection of all closed sets containing $A$.
\end{definition}

Casually, I like to think about it as the outer edge of the space, unioned with the space itself. For a closed space, like $[0,1]$, 
the closure is $\{0\} \cup \{1\}$, which is contained in the closed set. Note that all closed sets are equal to the closure of the
closed sets; this is because the smallest closed set containing a closed set $A$ is $A$ itself. 

\begin{notation}
  The closure of a set $A$ is denoted as $\bar{A}$.
\end{notation}

So for closed sets $A$, $\bar{A} = A$. Similarly, we have:

\begin{definition}
  The \textbf{interior} of $A$ is defined as the union of all open sets contained in $A$.
\end{definition}

\begin{notation}
  The interior of a set $A$ is denoted as $\text{Int} A$.
\end{notation}

For similar logic as before, for open sets $A$, $\text{Int} A = A$. 

\begin{definition}
  If $A$ is a subset of the topological space $X$ and if $x$ is a point of $X$. we say that $x$ is a
  \textbf{limit point/ cluster point/ point of accumulation} of $A$ if every neighborhood of $x$ intersects $A$ in some point other than 
  $x$ itself.
  \par You can also say $x$ is a limit point of $A$ if $x$ is in the closure of $A - \{ x \}$.
\end{definition}

A limit point is just a point on the boundary on a subspace $A$, although it doesn't necessarily have to be in $A$ itself. The 
limit points are kind of the boundary part of the closure. Hence the following theorem:

\begin{theorem}{}
  Let $A$ be a subset of the topological space $X$; let $A'$ be the set of all the limit points of $A$. Then 
  $$ \bar{A} = A \cup A'. $$
\end{theorem}

One can then consider the relationship between limit points and a set being closed. If the "boundary" of a set is 
made up of limit points, one can see that:

\begin{theorem}{}
  A subset of a topological space is closed if and only if it contains all its limit points.
\end{theorem}

\begin{definition}
  A collection $\mathcal{A}$ of subsets of a space $X$ is said to \textbf{cover} $X$, or to be a \textbf{covering} of $X$, if the union
  of the elements of $\mathcal{A}$ is equal to $X$. It is called an \textbf{open covering} of $X$ if its elements
  are open subsets of $X$.
\end{definition}

\begin{definition}
  A Space $X$ is said to be \textbf{compact} if every open covering $\mathcal{A}$ of $X$ contains a finite subcollection
  that also covers $X$. 
\end{definition}

\begin{definition}
  A topological space $X$ is called a \textbf{Hausdorff space} if for each pair $x_1,x_2$ of distinct points of $X$, 
  there exist neighborhoods $U_1$ and $U_2$ of $x_1$ and $x_2$ respectively that are disjoint.
\end{definition}

\begin{definition}
  Let $X$ be a topological space. A \textbf{separation} of $X$ is a pair $U, V$ of disjoint nonempty open subsets 
  of $X$ whose union is $X$. A space $X$ is said to be \textbf{connected} if there does not exist a separation 
  of $X$.
\end{definition}

This is really as simple as it sounds. Examples:

INSERT IPAD DRAWING HERE

\begin{definition}
  A space is called \textbf{path connected} if every pair of points in $X$ can be joined by a path in $X$
\end{definition}

\begin{uq}
  So how is there a difference between being path connected and connected? Shouldn't being connected, so that
  the space has no disjoint parts, be enough to say that a path can be drawn from point to point?
\end{uq}

\begin{answer}
  The biggest example is the \textbf{Topologists sine curve}:
  $$ y(x) = 
    \begin{cases}
      \sin\left( \frac{1}{x} \right) & \text{if } 0 < x < 1, \\
      0 & \text{if } x = 0,
    \end{cases} $$ 
  This is a curve that is connected, but \textbf{not} path connected.
  Proof of connectedness:

  \par I'm not exactly sure why this works, but my thoughts are just because you technically can't find any separation 
  between the two parts, because the limit of $\sin{\frac{1}{x}}$ as $x \to 0$ from the right is 0; however, because there 
  is not actually any connected, you cannot draw a path. At least that's what I'm getting from this right now. 
\end{answer}

\newpage

\section*{Algebraic Topology Definitions}

\begin{definition}
  A map is \textbf{nullhomotopic} if it is homotopic to a constant map.
\end{definition}

\begin{definition}
  A space is \textbf{contractible} if it is homotopically equivalent to a point. 
\end{definition}

So it's contractible if it can homotopically be squeezed into a point. 

\begin{theorem}
  A space is contractible if and only if it's identity map is nullhomotopic.
\end{theorem}
\begin{newproof}
  \begin{itemize}
    \item[($\implies$)] Assume a space $X$ is contractible, so that $X$ is homotopically equivalent to a point $x_0$. 
      Then there exists maps $f: X \to x_0$, $g:x_0 \to X$ such that $g \circ f \cong \text{id}_X$. But $g \circ f: X 
      \to X, x \mapsto x_0$, making it a constant map.
    \item[($\impliedby$)] If the identity map is nullhomotopic, it is homotopic to a constant map. This is 
      equivalent to saying there exists some $F(x,t)$ such that $F(x,0) = \text{id}_X$ and $F(x,1) = x_0 \forall x \in X$.
      This homotopy describes the space $X$ contracting to the point $x_0$ continuously.
  \end{itemize}
\end{newproof}

The backwards direction of this proof seems hand-wavey, so I may need to go back and look at this. 

\begin{definition}
  A space is called \textbf{simply connected} if it is path connected and has trivial fundamental group
\end{definition}

So a space is simply connected if there's nothing on the inside of it that causes loops to get "caught" on 
things. The first example that comes to my mind is how $S^2$ is simply connected, because it is path connected clearly, 
and any loop on $S^2$ can be contracted to a point (trivial fundamental group). On the other hand, the torus is path
connected, but a loop could be caught around the donut hole in the middle, unable to contract, resulting in a nontrivial 
fundamental group ($\Z \times \Z$, specifically).

\end{document}
