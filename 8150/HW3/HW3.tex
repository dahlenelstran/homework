\documentclass[12pt]{article}
 
\usepackage[margin=1in]{geometry} 
\usepackage{amsmath,amsthm,amssymb,outlines}
\usepackage{graphicx,tikzsymbols,tcolorbox}
\renewcommand\qedsymbol{$\blacksquare$}
\tcbuselibrary{most}

\newtcolorbox{newtitle}{
  enhanced,
  colframe=black,
  colback=white,
  boxsep=5pt,
  arc=8pt,
  sharp corners=south,
  borderline={0.5pt}{0pt}{black},
  borderline={1.8pt}{-5pt}{black},
  after skip=30pt
}

\newtcolorbox[auto counter]{statement}{
  enhanced,
  title={Exercise \thetcbcounter},
  colframe=black,
  colback=white,
  colbacktitle=white,
  fonttitle=\bfseries,
  coltitle=black,
  attach boxed title to top left={yshift=-0.25mm-\tcboxedtitleheight/2,yshifttext=2mm-\tcboxedtitleheight/2, xshift=2mm},
  boxed title style={boxrule=0.5mm}
}

\newtcolorbox{newproof}{
  enhanced,
  frame hidden,
  colback=white,
  title={Proof.},
  fonttitle=\bfseries,
  coltitle=black,
  colbacktitle=white,
  boxed title style={boxrule=0.5mm},
  attach boxed title to top left={yshift=-0.25mm-\tcboxedtitleheight/2,yshifttext=2mm-\tcboxedtitleheight/2, xshift=2mm},
  borderline west={1.5pt}{8pt}{black},
  after upper={\hfill $\blacksquare$}
}

\begin{document}

\begin{newtitle}
  \begin{center}
    \textbf{\Huge 8150 Homework III}
  \end{center}
  \textbf{Dahlen Elstran} \hfill \textbf{\today}
\end{newtitle}

\section*{Stein Problems}

\begin{statement}
  Using Euler's Formula
  $$ \sin \pi z = \frac{e^{i \pi z}-e^{-i \pi z}}{2i}, $$
  show that the complex zeroes of $\sin \pi z$ are exactly the integers, and that they are each of order 1. 
  \par Calculate the residue of $1/\sin \pi z$ at $z=n \in \mathbb{Z}$. 
\end{statement}
\begin{newproof}
  To find the zeroes, set Euler's Formula to zero:
  $$ \frac{e^{i \pi z}-e^{-i \pi z}}{2i} = 0. $$
  From here, we can find that 
  \begin{align*}
    e^{i \pi z} - e^{-i \pi z} &= 0 \\
    e^{i \pi z} &= e^{-i \pi z} \\ 
    i \pi z &= -i \pi z + 2 \pi i k , k \in \mathbb{Z} \\
    2i \pi z &= 2 i \pi k,
  \end{align*}
  so that $z = k$, meaning $z$ must be an integer. 
  \par To show that they are all of order one, we can find the derivate of $\sin \pi z$ to be 
  $ \pi \cos \pi z$, and notice that this is nonzero for any integer. 
  \par To find the residue of $1 / \sin \pi z$ at $z$ an integer, we can use this formula:
  $$ \text{Res}(f,a)=\lim_{z \to a} (z-a)(f(z)). $$
  So in our case, 
  $$ \text{Res}(\frac{1}{\sin \pi z}, n) = \lim_{z \to n} (z-n) \cdot \frac{1}{\sin \pi z}, $$
  where $n \in \mathbb{Z}$. 
  We know that near $z=n$, $\sin \pi z$ can be approximated using a Taylor Expansion, so that 
  $$ \frac{1}{\sin \pi z} \approx \frac{1}{\pi (z-n)(-1)^n}. $$
  Therefore we find that 
  $$ \text{Res}(\frac{1}{\sin \pi z}, n ) = \frac{1}{\pi (-1)^n}. $$
\end{newproof}

\begin{statement}
  Evaluate the inegral 
  $$ \int^{\infty}_{-\infty} \frac{dx}{1+x^4}. $$
  Where are the poles of $1/(1+z^4)$?
\end{statement}
\begin{newproof}
  
\end{newproof}

\section*{Tie Problems}

\begin{statement}
  Prove that if 
  $$ \sum^{\infty}_{n=-\infty} c_n(z-a)^n \text{ and } \sum^{\infty}_{n=-\infty} c'_n(z-a)^n $$
  are Laurent series expansions of $f(z)$, then $c_n=c'_n$ for all $n$.
\end{statement}
\begin{newproof}
  Let $f(z)=\sum^{\infty}_{n=-\infty} c_n(z-a)^n=\sum^{\infty}_{n=-\infty} c'_n(z-a)^n$. Then we know, for any integer $k$,  
  \begin{equation*}
    f(z)(z-a)^{-k-1}=\sum^{\infty}_{n=-\infty} c_n(z-a)^{n-k-1}=\sum^{\infty}_{n=-\infty} c'_n(z-a)^{n-k-1}
  \end{equation*}
  Then let $\gamma$ be any closed contour in the annulus going around $a$ once, and because it is a compact set 
  of points, the Luarent serieses can be integrated termwise:
  \begin{equation*}
    \sum^{\infty}_{n=-\infty}c_n \oint_{\gamma} (z-a)^{n-k-1}dz=\sum^{\infty}_{n=-\infty}c'_n \oint_{\gamma} (z-a)^{n-k-1}dz
  \end{equation*}
  We know that 
  \begin{equation*}
    \oint(z-a)^{n-k-1}dz = 2i\pi \text{ if } n=k \text{ and } 0 \text{ if } n \neq k
  \end{equation*}
  So then we are left with $2i\pi c_m = 21\pi c'_n$ for any $k$, which proves the statement.
\end{newproof}

\begin{statement}
  Expand $\frac{1}{1-z^2} + \frac{1}{3-z}$ in a series of the form $\sum^{\infty}{-\infty} a_nz^n$. How many 
  such expansions are there? In which domain is each of them valid?
\end{statement}
\begin{newproof}
  We find that:
  \begin{align*}
    \frac{1}{z-3} &= -\frac{1}{3} \frac{1}{1-3z^{-1}}=-\frac{1}{3} \sum_{ k \geq 0} 3^{-k}z^{k} \text{ for } \vert z \vert < 3 \\
                  &= \frac{1}{z} \frac{1}{1-3z^{-1}} = z^{-1}\sum_{k \geq 0} 3^kz^{-k} \text{ for } \vert z \vert > 3 
  \end{align*}
  and:
  \begin{align*}
    \frac{1}{1-z^2} &= \sum_{k \geq 0} z^{2k} \text{ for } \vert z \vert < 1 \\
                    &= \frac{1}{z^2} \frac{-1}{1-z^{-2}}=-z^{-2}\sum_{k \geq 0} z^{-2k} \text{ for } \vert z \vert > 1 
  \end{align*}
  So we can just list all the possible combinations to find:
  \begin{align*}
    f(z) &= -\frac{1}{3} \sum_{ k \geq 0} 3^{-k}z^{k} + \sum_{k \geq 0} z^{2k} \text{ for } \vert z \vert \in (-\infty, 1) \\
    f(z) &= -\frac{1}{3} \sum_{ k \geq 0} 3^{-k}z^{k}-z^{-2}\sum_{k \geq 0} z^{-2k} \text{ for } \vert z \vert \in (1,3) \\
    f(z) &= z^{-1}\sum_{k \geq 0} 3^kz^{-k} + \sum_{k \geq 0} z^{2k} \text{ for } \vert z \vert \in (-\infty,1) \cup (3, \infty) \\
    f(z) &= z^{-1}\sum_{k \geq 0} 3^kz^{-k}-z^{-2}\sum_{k \geq 0} z^{-2k} \text{ for } \vert z \vert \in (3,\infty)
  \end{align*}
\end{newproof}

\begin{statement}
  Let $P(z)$ and $Q(z)$ be polynomials with no common zeros. Assume $Q(a)=0$. Find the principal part of 
  $P(z)/Q(z)$ at $z=a$ if the zero $a$ is (i) simple; (ii) double. Express your answers explicitly using $P$ and $Q$. 
\end{statement}
\begin{newproof}
  \begin{itemize}
    \item[i.]
    \item[ii.]
  \end{itemize}
\end{newproof}

\begin{statement}
  Let $f(z)$ be a non-constant analytic function in $\vert z \vert > 0$ such that $f(z_n)=0$ for infinite 
  many points $z_n$ with $lim_{n \to \infty}z_n=0$. Show that $z=0$ is an essential singularity for $f(z)$.
\end{statement}
\begin{newproof}
  Assume, for contradiction, that $z=0$ is a removable singularity. Then $f$ would extend to a holomorphic 
  function over $z=0$, so that $f(0)=f(\lim z_n)=\lim f(z_n) = 0$. But then $f$ would have to be identically 
  zero, because of the identity principal. This contradicts the fact that $f$ is stated to be non-constant.
  \par Then assume for contradiction that $z=0$ is a pole. Then $f(z_n) \to \infty$. This is a contradiction
  because $f(z_n)=0$ infinitely many times.
  \par Thus $z=0$ must be an essential singularity. 
\end{newproof}

\begin{statement}
  Let $f$ be entire and suppose that $\lim_{x \to \infty} f(z) = \infty$. Show that $f$ is a polynomial.
\end{statement}
\begin{newproof}
  First, note that because $f$ is unbounded, there must exist some $R$ such that $f(D^c_R) \subset D^c$. 
  Therefore we know that $f$ is nonvanishing on $D^c_R$. Then we know the zeroes of $f$, $Z_f$, is a 
  closed subset of a compact set. Therefore we know it is either finite, or has an accumulation point.
  If it had an accumulation point, $f$ would have to be identically zero, so $Z_f$ must be finite. We can 
  then define, where $n$ represents the number of zeroes for $f$, 
  \begin{equation*}
    \phi (z) = \Pi_{i \leq n} (z-z_i) \text{ and } F(z)= \frac{\phi (z)}{f(z)}.
  \end{equation*}
  Then note that $F$ is nonvanishing, entire, and bounded. Thus by Liouville, it has to be constant, so 
  $f(z)=c \phi(z)$.
\end{newproof}

\begin{statement}
  Find the number of roots of $z^4 - 6z+3=0$ in $\vert z \vert < 1$ and $1 < \vert z \vert < 2$ respectively. 
\end{statement}
\begin{newproof}
  \begin{itemize}
    \item In $\vert z \vert < 1$:
      \subitem Small: $z^4 + 3$
      \subitem Big: $-6z$
    \item In $\vert z \vert = 1$:
      \subitem $\vert m(z) \vert = \vert z^4 + 3 \vert \leq \vert 4 \vert^4 + 3 = 4 < 6 = \vert -6z \vert = \vert M(z) \vert$
    \item In $\vert z \vert < 2$:
      \subitem Small: $-6z+3$
      \subitem Big: $z^4$
    \item In $\vert z \vert = 2$:
      \subitem $\vert m(z) \vert = \vert -6z+3 \vert \leq 6+3=9 < 2^4 = \vert M(z) \vert$
  \end{itemize}
  Therefore there is 1 root in $\vert z \vert < 1$, and there are 3 zeroes in $1 < \vert z \vert < 2$.
\end{newproof}

\begin{statement}
  Prove that $z^4+2z^3-2z+10=0$ has exactly one root in each open quadrant. 
\end{statement}
\begin{newproof}
  First note that it is sufficient to prove the existence of exactly one root in $Q_1$, because conjugate 
  pairs proves the existence in the other open quadrant. We know the polynomial is entire, so we can use the 
  argument principle to count the zeroes. Let $\gamma$ be made up of 
  \begin{align*}
    \gamma_1 &= [0,R] \\
    \gamma_2 &= Re^{it} \text{ for } t \in [0, \pi / 2] \\
    \gamma_ 3 &= i[0,R].
  \end{align*}
  Then we can consider 
  \begin{equation*}
    Z_f = \frac{1}{2\pi i} \int_{\gamma} \partial^{\log} f(z)dz = \Delta_{\gamma}\text{Arg}(f).
  \end{equation*}
  Then for each part of gamma, 
  \begin{align*}
    \Delta_{\gamma_1}\text{Arg}(f) &= 0 \\
    \Delta_{\gamma_2}\text{Arg}(f) &= 4(\frac{\pi}{2}) = 2\pi \\
    \Delta_{\gamma_3}\text{Arg}(f) &=0.
  \end{align*}
  To prove the last part, consider $f(it)=t^4-it^3-2it+10=t^4(1-it^{-1}-2it^{-2}+10t^{-4})$.
  \par Thus $\Delta_{\gamma}\text{Arg}(f)=1$, so as $R \to \infty$, there is only 1 zero. 
\end{newproof}

\end{document}
