\documentclass[12pt]{article}
 
\usepackage[margin=1in]{geometry} 
\usepackage{amsmath,amsthm,amssymb,outlines}
\usepackage{graphicx,tikzsymbols,tcolorbox}
\renewcommand\qedsymbol{$\blacksquare$}
\tcbuselibrary{most}

\newtcolorbox{newtitle}{
  enhanced,
  colframe=black,
  colback=white,
  boxsep=5pt,
  arc=8pt,
  sharp corners=south,
  borderline={0.5pt}{0pt}{black},
  borderline={1.8pt}{-5pt}{black},
  after skip=30pt
}

\newtcolorbox[auto counter]{statement}{
  enhanced,
  title={Exercise \thetcbcounter},
  colframe=black,
  colback=white,
  colbacktitle=white,
  fonttitle=\bfseries,
  coltitle=black,
  attach boxed title to top left={yshift=-0.25mm-\tcboxedtitleheight/2,yshifttext=2mm-\tcboxedtitleheight/2, xshift=2mm},
  boxed title style={boxrule=0.5mm}
}

\newtcolorbox{newproof}{
  enhanced,
  frame hidden,
  colback=white,
  title={Proof.},
  fonttitle=\bfseries,
  coltitle=black,
  colbacktitle=white,
  boxed title style={boxrule=0.5mm},
  attach boxed title to top left={yshift=-0.25mm-\tcboxedtitleheight/2,yshifttext=2mm-\tcboxedtitleheight/2, xshift=2mm},
  borderline west={1.5pt}{8pt}{black},
  after upper={\hfill $\blacksquare$}
}

\begin{document}

\begin{newtitle}
  \begin{center}
    \textbf{\Huge 8150 Homework III}
  \end{center}
  \textbf{Dahlen Elstran} \hfill \textbf{\today}
\end{newtitle}

\section*{Stein Problems}

\begin{statement}

\end{statement}


\section*{Tie Problems}

\begin{statement}
  Prove that if 
  $$ \sum^{\infty}_{n=-\infty} c_n(z-a)^n \text{ and } \sum^{\infty}_{n=-\infty} c'_n(z-a)^n $$
  are Laurent series expansions of $f(z)$, then $c_n=c'_n$ for all $n$.
\end{statement}
\begin{newproof}
  Let $f(z)=\sum^{\infty}_{n=-\infty} c_n(z-a)^n=\sum^{\infty}_{n=-\infty} c'_n(z-a)^n$. Then we know, for any integer $k$,  
  \begin{equation*}
    f(z)(z-a)^{-k-1}=\sum^{\infty}_{n=-\infty} c_n(z-a)^{n-k-1}=\sum^{\infty}_{n=-\infty} c'_n(z-a)^{n-k-1}
  \end{equation*}
  Then let $\gamma$ be any closed contour in the annulus going around $a$ once, and because it is a compact set 
  of points, the Luarent serieses can be integrated termwise:
  \begin{equation*}
    \sum^{\infty}_{n=-\infty}c_n \oint_{\gamma} (z-a)^{n-k-1}dz=\sum^{\infty}_{n=-\infty}c'_n \oint_{\gamma} (z-a)^{n-k-1}dz
  \end{equation*}
  We know that 
  \begin{equation*}
    \oint(z-a)^{n-k-1}dz = 2i\pi \text{ if } n=k \text{ and } 0 \text{ if } n \neq k
  \end{equation*}
  So then we are left with $2i\pi c_m = 21\pi c'_n$ for any $k$, which proves the statement.
\end{newproof}

\begin{statement}
  Expand $\frac{1}{1-z^2} + \frac{1}{3-z}$ in a series of the form $\sum^{\infty}{-\infty} a_nz^n$. How many 
  such expansions are there? In which domain is each of them valid?
\end{statement}
\begin{newproof}
  We find that:
  \begin{align*}
    \frac{1}{z-3} &= -\frac{1}{3} \frac{1}{1-3z^{-1}}=-\frac{1}{3} \sum_{ k \geq 0} 3^{-k}z^{k} \text{ for } \vert z \vert < 3 \\
                  &= \frac{1}{z} \frac{1}{1-3z^{-1}} = z^{-1}\sum_{k \geq 0} 3^kz^{-k} \text{ for } \vert z \vert > 3 
  \end{align*}
  and:
  \begin{align*}
    \frac{1}{1-z^2} &= \sum_{k \geq 0} z^{2k} \text{ for } \vert z \vert < 1 \\
                    &= \frac{1}{z^2} \frac{-1}{1-z^{-2}}=-z^{-2}\sum_{k \geq 0} z^{-2k} \text{ for } \vert z \vert > 1 
  \end{align*}
  So we can just list all the possible combinations to find:
  \begin{align*}
    f(z) &= -\frac{1}{3} \sum_{ k \geq 0} 3^{-k}z^{k} + \sum_{k \geq 0} z^{2k} \text{ for } \vert z \vert \in (-\infty, 1) \\
    f(z) &= -\frac{1}{3} \sum_{ k \geq 0} 3^{-k}z^{k}-z^{-2}\sum_{k \geq 0} z^{-2k} \text{ for } \vert z \vert \in (1,3) \\
    f(z) &= z^{-1}\sum_{k \geq 0} 3^kz^{-k} + \sum_{k \geq 0} z^{2k} \text{ for } \vert z \vert \in (-\infty,1) \cup (3, \infty) \\
    f(z) &= z^{-1}\sum_{k \geq 0} 3^kz^{-k}-z^{-2}\sum_{k \geq 0} z^{-2k} \text{ for } \vert z \vert \in (3,\infty)
  \end{align*}
\end{newproof}

\begin{statement}
  Let $P(z)$ and $Q(z)$ be polynomials with no common zeros. Assume $Q(a)=0$. Find the principal part of 
  $P(z)/Q(z)$ at $z=a$ if the zero $a$ is (i) simple; (ii) double. Express your answers explicitly using $P$ and $Q$. 
\end{statement}
\begin{newproof}
  \begin{itemize}
    \item[i.]
    \item[ii.]
  \end{itemize}
\end{newproof}

\end{document}
