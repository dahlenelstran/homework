\documentclass[12pt]{article}
 
\usepackage[margin=.5in]{geometry} 
\usepackage{amsmath,amsthm,amssymb,outlines}
\usepackage{graphicx,tikzsymbols,tcolorbox}
\renewcommand\qedsymbol{$\blacksquare$}
\newcommand{\Z}{\mathbb{Z}}
\newcommand{\R}{\mathbb{R}}
\tcbuselibrary{most}

\newtcolorbox{newtitle}{
  enhanced,
  colframe=black,
  colback=white,
  boxsep=5pt,
  arc=8pt,
  sharp corners=south,
  borderline={0.5pt}{0pt}{black},
  borderline={1.8pt}{-5pt}{black},
  after skip=30pt
}

\newtcolorbox[auto counter]{statement}[1][]{
  enhanced,
  title={Exercise \ifx\\#1\\\thetcbcounter\else#1\fi},
  colframe=black,
  colback=white,
  colbacktitle=white,
  fonttitle=\bfseries,
  coltitle=black,
  attach boxed title to top left={yshift=-0.25mm-\tcboxedtitleheight/2,yshifttext=2mm-\tcboxedtitleheight/2, xshift=2mm},
  boxed title style={boxrule=0.5mm}
}

\newtcolorbox{newproof}{
  enhanced,
  breakable,
  frame hidden,
  colback=white,
  title={Proof.},
  fonttitle=\bfseries,
  coltitle=black,
  colbacktitle=white,
  boxed title style={boxrule=0.5mm},
  attach boxed title to top left={yshift=-0.25mm-\tcboxedtitleheight/2,yshifttext=2mm-\tcboxedtitleheight/2, xshift=2mm},
  borderline west={1.5pt}{8pt}{black},
  after upper={\hfill $\blacksquare$}
}

\begin{document}

\begin{newtitle}
  \begin{center}
    \textbf{\Huge Homework 8}
  \end{center}
  \textbf{} \hfill \textbf{\today}
\end{newtitle}

\begin{statement}
  Determine whether there exists a short exact sequence $0 \to \Z_4 \to \Z_8 \bigoplus \Z_2 \to \Z_4 \to 0$.
  More generally, determine which abelian groups $A$ fit into a short exact sequence 
  $0 \to \Z_{p^m} \to A \to \Z_{p^n} \to 0$ with $p$ prime. What about the case of short exact 
  sequences $0 \to \Z \to A \to \Z_n \to 0$?
\end{statement}
\begin{newproof}
    \begin{itemize}
        \item For an exact sequence $0 \to \Z_4 \to \Z_8 \bigoplus \Z_2 \to \Z_4 \to 0$, we know $f: \Z_4 \to \Z_8 \bigoplus \Z_2$ is injective, and $g: \bigoplus \Z_2 \to \Z_4$ is surjective. Because $f$ is injective, we know im$f \cong \Z_4$, so ker$g \cong  \Z_4$. By the first isomorphism theorem, because $g$ is surjective, we know $\Z_4 \cong \Z_8 \bigoplus \Z_2 / \text{ker}f = \Z_8 \bigoplus \Z_2 / \Z_4$. But $\Z_4$ isn't a subgroup of $\Z_2$, so we're really just modding out from $\Z_8$, so we are left with 
        $$ \Z_4 \cong \Z_8 \bigoplus \Z_2 / \Z_4 =  \Z_2 \bigoplus \Z_2$$
        which is clearly false. Thus, there next exists such an exact sequence. 
        \item From a similar setup to last time, the condition that must be met is 
        $$\Z_{p^n} \cong A / \Z_{p^m}$$
        So $A$ must be isomorphic to $\Z_{p^{n+m}}$, and also must have a subgroup isomorphic to $\Z_{p^m}$.
        \item Using the above part, $A$ must contain $Z$ as a subgroup with index $n$:
            \begin{itemize}
                \item $\Z$ (when $n=1$)
                \item $\Z \oplus \Z_{d_1} \oplus \cdots \oplus \Z_{d_k}$ where $d_1 \cdots d_k = n$
            \end{itemize}
    \end{itemize}
\end{newproof}

\begin{statement}
    \begin{itemize}
        \item Show that $H_0(X,A) = 0$ iff $A$ meets each path-component of $X$. 
        \item Show that $H_1(X,A) = 0$ iff $H_1(A) \to H_1(X)$ is surjective and each path-component of $X$ contains at most one path-component of $A$. 
    \end{itemize}
\end{statement}
\begin{newproof}
\begin{itemize}
    \item The long exact sequence for \((X,A)\) gives:
    \[
    H_0(A) \xrightarrow{i_*} H_0(X) \to H_0(X,A) \to 0.
    \]
    $(\implies)$ If \( H_0(X,A) = 0 \), then \( i_* \) is surjective, so \( A \) meets every path-component of \( X \). \\
    $(\impliedby)$ Conversely, if \( A \) meets every path-component, \( i_* \) is surjective, implying \( H_0(X,A) = 0 \).

    \item From the long exact sequence:
    \[
    H_1(A) \xrightarrow{i_*} H_1(X) \to H_1(X,A) \to H_0(A) \xrightarrow{i_*} H_0(X).
    \]
    $(\implies)$ If \( H_1(X,A) = 0 \), then:
    \begin{itemize}
        \item \( i_*: H_1(A) \to H_1(X) \) is surjective, and
        \item \( i_*: H_0(A) \to H_0(X) \) is injective.
    \end{itemize}
    $(\impliedby)$ If the conditions hold, surjectivity of \( H_1(A) \to H_1(X) \) forces \( H_1(X,A) = 0 \), and injectivity of \( H_0(A) \to H_0(X) \) follows from (ii).
\end{itemize}

\end{newproof}

\begin{statement}
    Show that $\tilde{H}_n(X) \cong \tilde{H}_{n+1}(SX)$ for all $n$, where $SX$ is the suspension of $X$. More generally, thinking of $SX$ as the union of two cones $CX$ with their bases identified, compute the reduced homology groups of the union of any finite number of cones $CX$ with their bases identified. 
\end{statement}
\begin{newproof}
    \begin{itemize}
        \item Consider the suspension $SX$ with $U = SX \setminus \{\text{north pole}\}$ and $V = SX \setminus \{\text{south pole}\}$, so that $U \cap V \simeq X$ and $U,V$ contractible. 

        The long exact sequence for reduced homology of the pair $(SX, X)$ gives:
        $$ \cdots \to \tilde{H}_{n+1}(SX) \to \tilde{H}_{n+1}(SX/X) \to \tilde{H}_n(X) \to \tilde{H}_n(SX) \to \cdots $$
        Since $SX/X$ is a wedge of two spheres, and $\tilde{H}_n(SX) \cong \tilde{H}_n(CX_+) \oplus \tilde{H}_n(CX_-) = 0$ (cones are contractible), we get:
        $$ 0 \to \tilde{H}_{n+1}(SX) \to \tilde{H}_{n+1}(S^1 \vee S^1) \to \tilde{H}_n(X) \to 0 $$
        From this short exact sequence, we find that $\tilde{H}_n(X) \cong \tilde{H}_{n+1}(SX)$.

        \item 
\end{itemize}
\end{newproof}

\begin{statement}
    Making the preceding problem more concrete, construct explicit chain maps $s: C_n(X) \to C_{n+1}(SX)$ inducing isomorphisms $\tilde{H}_n(X) \to \tilde{H}_{n+1}(SX)$.
\end{statement}
\begin{newproof}

\end{newproof}

\begin{statement}
    Prove by induction on dimension the following facts about the homology of a finite-dimensional CW complex $X$, using the observation that $X^n/X^{n-1}$ is a wedge sum of $n$-spheres:
    \begin{itemize}
        \item If (X) has dimension $n$ then $H_i(X)=0$ for $i > n$ and $H_n(X)$ is free 
        \item $H_n(X)$ is free with basis in bijective correspondence with the $n$-cells if there are no cells of dimension $n-1$ or $n+1$.
        \item If $X$ has $k$ $n$-cells, then $H_n(X)$ is generated by at most $k$ elements.
    \end{itemize}
\end{statement}
\begin{newproof}
    \begin{itemize}
        \item \textit{Base case:} For $X^0$ (0-dimensional complex), $H_i(X^0) = 0$ for $i > 0$ and $H_0(X^0)$ is free abelian on the path        components. \\
        \textit{Inductive step:} Assume true for $(n-1)$-dimensional complexes. Consider the long exact sequence for $(X^n, X^{n-1})$:
            $$\cdots \to H_{i+1}(X^n/X^{n-1}) \to H_i(X^{n-1}) \to H_i(X^n) \to H_i(X^n/X^{n-1}) \to \cdots$$
            Since $X^n/X^{n-1}$ is a wedge of $n$-spheres, $H_i(X^n/X^{n-1}) = 0$ for $i \neq n$. For $i > n$, we get:
            $$0 \to H_i(X^{n-1}) \to H_i(X^n) \to 0$$
            By induction, $H_i(X^{n-1}) = 0$, so $H_i(X^n) = 0$. For $i = n$, the sequence gives:
            $$0 \to H_n(X^n) \to H_n(X^n/X^{n-1}) \to H_{n-1}(X^{n-1})$$
            Since $H_n(X^n/X^{n-1})$ is free and $H_n(X^n)$ is its subgroup, $H_n(X^n)$ is free.
            
        \item When there are no $(n-1)$-cells or $(n+1)$-cells, the exact sequence becomes:
            $$0 \to H_n(X^n) \to \mathbb{Z}^k \to 0$$
            where $k$ is the number of $n$-cells, because $X^{n-1}=X^{n-2}$. This gives an isomorphism $H_n(X) \cong \mathbb{Z}^k$, with basis corresponding to the $n$-cells. The absence of $(n-1)$-cells means no relations come from below, and no $(n+1)$-cells means no relations are added from above.

        \item Let $X$ have $k$ $n$-cells:
            $$0 \to \text{im}(H_n(X^n)) \to \mathbb{Z}^k \to H_{n-1}(X^{n-1})$$
            Thus $H_n(X^n)$ is isomorphic to a subgroup of $\mathbb{Z}^k$, which is free abelian of rank $\leq k$. Even when $(n+1)$-cells are present, they can only make this subgroup smaller, so $H_n(X)$ must be generated by $\leq k$ elements.
    \end{itemize}
\end{newproof}

\end{document}
