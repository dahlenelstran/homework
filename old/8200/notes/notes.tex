\documentclass[12pt]{article}
 \usepackage[margin=1in]{geometry} 
\usepackage{amsmath,amsthm,amssymb,outlines}
\usepackage{graphicx,tikzsymbols,tcolorbox}
\newcommand{\C}{\mathbb{C}}
\renewcommand\qedsymbol{$\blacksquare$}
\tcbuselibrary{most}

% Defining new environments

\newtcolorbox{newtitle}{
  enhanced,
  colframe=black,
  colback=white,
  boxsep=5pt,
  arc=8pt,
  sharp corners=south,
  borderline={0.5pt}{0pt}{black},
  borderline={1.8pt}{-5pt}{black},
  after skip=30pt
}

\newtcolorbox[auto counter, number within=section]{theorem}[2][title]{
  enhanced,
  title={Theorem \thetcbcounter \if\relax\detokenize{#2}\relax\else\ (#2)\fi},
  colframe=black,
  colback=white,
  colbacktitle=white,
  fonttitle=\bfseries,
  coltitle=black,
  attach boxed title to top left={yshift=-0.25mm-\tcboxedtitleheight/2,yshifttext=2mm-\tcboxedtitleheight/2, xshift=2mm},
  boxed title style={boxrule=0.5mm}
}

\newtcolorbox[auto counter, number within=section]{definition}{
  enhanced,
  title={Definition \thetcbcounter},
  colframe=black,
  colback=white,
  colbacktitle=white,
  fonttitle=\bfseries,
  coltitle=black,
  attach boxed title to top left={yshift=-0.25mm-\tcboxedtitleheight/2,yshifttext=2mm-\tcboxedtitleheight/2, xshift=2mm},
  boxed title style={boxrule=0.5mm}
}

\newtcolorbox[auto counter, number within=section]{fact}{
  enhanced,
  title={Fact \thetcbcounter},
  colframe=black,
  colback=white,
  colbacktitle=white,
  fonttitle=\bfseries,
  coltitle=black,
  attach boxed title to top left={yshift=-0.25mm-\tcboxedtitleheight/2,yshifttext=2mm-\tcboxedtitleheight/2, xshift=2mm},
  boxed title style={boxrule=0.5mm}
}

\newtcolorbox[auto counter, number within=section]{notation}{
  enhanced,
  title={Notation},
  colframe=black,
  colback=white,
  colbacktitle=white,
  fonttitle=\bfseries,
  coltitle=black,
  attach boxed title to top left={yshift=-0.25mm-\tcboxedtitleheight/2,yshifttext=2mm-\tcboxedtitleheight/2, xshift=2mm},
  boxed title style={boxrule=0.5mm}
}

\newtcolorbox{newproof}{
  enhanced,
  frame hidden,
  colback=white,
  title={Proof.},
  fonttitle=\bfseries,
  coltitle=black,
  colbacktitle=white,
  boxed title style={boxrule=0.5mm},
  attach boxed title to top left={yshift=-0.25mm-\tcboxedtitleheight/2,yshifttext=2mm-\tcboxedtitleheight/2, xshift=2mm},
  borderline west={1.5pt}{8pt}{black},
  after upper={\hfill $\blacksquare$},
  after skip=30pt
}

\newtcolorbox[auto counter, number within=section]{uq}{
  enhanced,
  title=Question,
  colframe=red,
  colback=white,
  colbacktitle=white,
  fonttitle=\bfseries,
  coltitle=black,
  attach boxed title to top left={yshift=-0.25mm-\tcboxedtitleheight/2,yshifttext=2mm-\tcboxedtitleheight/2, xshift=2mm},
  boxed title style={boxrule=0.5mm}
}

\newtcolorbox[auto counter, number within=section]{aq}{
  enhanced,
  title=Question,
  colframe=green!50!black,
  colback=white,
  colbacktitle=white,
  fonttitle=\bfseries,
  coltitle=black,
  attach boxed title to top left={yshift=-0.25mm-\tcboxedtitleheight/2,yshifttext=2mm-\tcboxedtitleheight/2, xshift=2mm},
  boxed title style={boxrule=0.5mm}
}

\newtcolorbox{answer}{
  enhanced,
  frame hidden,
  colback=white,
  colframe=green!50!black,
  title={Answer},
  fonttitle=\bfseries,
  coltitle=black,
  colbacktitle=white,
  boxed title style={boxrule=0.5mm},
  attach boxed title to top left={yshift=-0.25mm-\tcboxedtitleheight/2,yshifttext=2mm-\tcboxedtitleheight/2, xshift=2mm},
  borderline west={1.5pt}{8pt}{green!50!black}
}

% Redefine section as well

% Renew commands later

\begin{document}

\subsection{Basic Constructions}

\textbf{REORDER THIS}

A simple distinction between being homeomorphic, homotopic, and homotopy equivalent:
\begin{itemize}
  \item A \textbf{homeomorphism} is the strongest of the three-- two spaces being homeomorphic means that 
    there is a bijctive coorespondance between the spaces themselves, and between the open sets of the 
    spaces. For two spaces to be homeomorphic, there must exist a function $f$ between them such that 
    \subitem $f$ is continuous
    \subitem $f$ is bijective
    \subitem the inverse function $f^{-1}$ is also continuous
  \item A \textbf{homotopy} is a continuous deformation between two continuous functions. 
\end{itemize}

\subsection{Van Kampen's Theorem}

\subsection{Covering Spaces}

\begin{definition}
  A \textbf{covering space} of a space $X$ is a space $\tilde{X}$ together with a map $\tilde{p} \tilde{X} \to X$ 
  satisfying the following condition: Each point $x \in X$ has an open neighborhood $U$ in $X$ such that 
  $p^{-1}(U)$ is a union of disjoint open sets in $\tilde{X}$, each of which is mapped homeomorphically 
  onto $U$ by $p$.
\end{definition}

\begin{aq}
  What does it mean for something to be mapped homeomorphically?
\end{aq}

\begin{answer}
  All it means is that $p$ is a homeomorphism-- which, if you'll recall from the beginning of the class, 
  just means that it is continuous, bijective, and has a continuous inverse. 
\end{answer}

All this is saying is that there is a space such that 

\section{Homology}

\subsection{Simplicial and Singular Homology}

\end{document}
