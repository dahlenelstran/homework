\documentclass[12pt]{article}
 
\usepackage[margin=.5in]{geometry} 
\usepackage{amsmath,amsthm,amssymb,outlines}
\usepackage{graphicx,tikzsymbols,tcolorbox, bbm}
\renewcommand\qedsymbol{$\blacksquare$}
\newcommand{\Z}{\mathbb{Z}}
\newcommand{\R}{\mathbb{R}}
\tcbuselibrary{most}

\newtcolorbox{newtitle}{
  enhanced,
  colframe=black,
  colback=white,
  boxsep=5pt,
  arc=8pt,
  sharp corners=south,
  borderline={0.5pt}{0pt}{black},
  borderline={1.8pt}{-5pt}{black},
  after skip=30pt
}

\newtcolorbox[auto counter]{statement}[1][]{
  enhanced,
  breakable,
  title={Exercise \ifx\\#1\\\thetcbcounter\else#1\fi},
  colframe=black,
  colback=white,
  colbacktitle=white,
  fonttitle=\bfseries,
  coltitle=black,
  attach boxed title to top left={yshift=-0.25mm-\tcboxedtitleheight/2,yshifttext=2mm-\tcboxedtitleheight/2, xshift=2mm},
  boxed title style={boxrule=0.5mm}
}

\newtcolorbox{newproof}{
  enhanced,
  breakable,
  frame hidden,
  colback=white,
  title={Proof.},
  fonttitle=\bfseries,
  coltitle=black,
  colbacktitle=white,
  boxed title style={boxrule=0.5mm},
  attach boxed title to top left={yshift=-0.25mm-\tcboxedtitleheight/2,yshifttext=2mm-\tcboxedtitleheight/2, xshift=2mm},
  borderline west={1.5pt}{8pt}{black},
  after upper={\hfill $\blacksquare$}
}

\begin{document}

\begin{newtitle}
  \begin{center}
    \textbf{\Huge 8200 Homework 9}
  \end{center}
  \textbf{} \hfill \textbf{\today}
\end{newtitle}

\begin{statement}[1]
  Show that $S^1 \times S^1$ and $S^1 \vee S^1 \vee S^2$ has isomorphic homology groups in all dimensions, but thier universal 
  covering spaces do not.
\end{statement}
\begin{newproof}
  We know the torus $S^1 \times S^1$ is a CW complex with one 0-cell $e^0$, two 1-cells $e^1_a, e^1_b$, and one 2-cell $e^2$.
  Then  we know:
  \begin{itemize}
    \item $H_0(S^1 \times S^1)$ counts path components. Since the torus is connected, $H_0 \cong \mathbb{Z}$.
    \item $H_1(S^1 \times S^1)$ can be computed by looking at the boundary maps. The only possible nontrivial boundary map involving 1-cells 
      is $\partial_1: C_1 \to C_0$. However, the single 0-cell means that the boundary of each 1-cell is that same 0-cell. 
      Therefore we end up with two independent loops, so $H_1 \cong \mathbb{Z}^2$.
    \item For $H_2(S^1 \times S^1)$, the boundary of the 2-cell corresponds to a loop $aba^{-1}b^{-1}$ in the 1-skeleton that is 
      null-homotopic in the 1-skeleton. This ensures that the resulting $\partial_2$ map is zero in homology, so 
      $H_2 \cong \mathbb{Z}$. 
    \item In higher dimensions ($n>2$), there are no $n$-cells, so $H_n(S^1 \times S^1) = 0$ for $n>2$.
  \end{itemize}

  Therefore
  $$ H_n(S^1 \times S^1) \;\cong\;
    \begin{cases}
      \mathbb{Z} & n=0,\\
      \mathbb{Z}^2 & n=1,\\
      \mathbb{Z} & n=2,\\
      0 & \text{otherwise}.
    \end{cases} $$

  For $Y = S^1 \vee S^1 \vee S^2$, we can use the following fact about wedge sums: $$ \widetilde{H}_n(X \vee Z) \;\cong\; 
    \widetilde{H}_n(X) \;\oplus\; \widetilde{H}_n(Z) \quad \text{for all } n\ge1. $$

  Let $X = S^1 \vee S^1$ and $Z = S^2$. Then
  $$ \widetilde{H}_n(Y) \;=\; \widetilde{H}_n(X \vee Z) 
    \;\cong\; 
    \widetilde{H}_n(X) \;\oplus\; \widetilde{H}_n(Z). $$
  We know that 
  $$ \widetilde{H}_n(S^1) = 
    \begin{cases}
      \mathbb{Z} & n=1,\\
      0 & \text{otherwise},
    \end{cases}
    \quad
    \widetilde{H}_n(S^2) =
      \begin{cases}
        \mathbb{Z} & n=2,\\
        0 & \text{otherwise},
      \end{cases}
      \quad
      \widetilde{H}_n(S^1 \vee S^1) \cong \widetilde{H}_n(S^1) \oplus \widetilde{H}_n(S^1). $$
  Therefore:
  $$ \widetilde{H}_n(Y) 
      \;=\;
    \widetilde{H}_n(X \vee Z)
    \;=\;
    \begin{cases}
      \mathbb{Z}^2 \oplus 0 = \mathbb{Z}^2 & n=1,\\
      0 \oplus \mathbb{Z} = \mathbb{Z} & n=2,\\
      0 & \text{otherwise}.
    \end{cases} $$
  So that
  $$ H_n(S^1 \vee S^1 \vee S^2) 
    \;\cong\;
    \begin{cases}
      \mathbb{Z} & n=0,\\
      \mathbb{Z}^2 & n=1,\\
      \mathbb{Z} & n=2,\\
      0 & \text{otherwise}.
    \end{cases} $$
  Therefore $H_n(S^1 \times S^1) \;\cong\; H_n(S^1 \vee S^1 \vee S^2).$

  For the universal covering spaces, we know that the torus's covering space is $\R^2$, and for $S^1 \vee S^1 \vee S^2$, we can use an infinite tree with a 2-sphere at each vertex, which is clearly not homeomorphic to $\mathbb{R}^2$.
\end{newproof}

\begin{statement}[2]
  Show that for every $f:S^n \to S^n$, degree of $Sf:S^{n+1} \to S^{n+1}$ is equal to degree of $f$. Here, $Sf$ denotes the suspension of $f$ 
  which is the map induced from $f \times$id$:S^n \times [0,1] \to S^n \times [0,1]$ on $SS^n \cong S^{n+1}$. 
\end{statement}
\begin{newproof}
  The degree of a map $g: S^k \to S^k$ can be characterized by its induced map 
      $$
        g_*: H_k(S^k) \;\longrightarrow\; H_k(S^k).
      $$
      Since $H_k(S^k) \cong \mathbb{Z}$, the map $g_*$ must be multiplication by some integer $d$. In passing from $f$ to $Sf$, $H_{n+1}(S^{n+1})$ is essentially determined by how $f$ acts on $H_n(S^n)$. This is because when the suspension is done, collapsing $S^n \times \{0\}$ and $S^n \times \{1\}$ to points does not alter the $d$ integer.
    
    Therefore, $(Sf)_* : H_{n+1}(S^{n+1}) \to H_{n+1}(S^{n+1})$ acts on the generator $[S^{n+1}]$ by the same integer $\deg(f)$. Therefore,
      $$
        \deg(Sf) \;=\; \deg(f).
      $$
\end{newproof}


\begin{statement}[3]
  Given a map $f: S^{2n} \to S^{2n}$, show that there is some point $x \in S^{2n}$ with either $f(x)=x$ or $f(x)=-x$. Deduce that 
  every map $\mathbb{R}P^{2n} \to \mathbb{R}P^{2n}$ has a fixed point. Construct maps $\mathbb{R}P^{2n-1} \to \mathbb{R}P^{2n-1}$
  without fixed points from linear transformations $\mathbb{R}^{2n} \to \mathbb{R}^{2n}$ without eigenvectors. 
\end{statement}
\begin{newproof}
    \begin{itemize}
        \item Suppose, for contradiction, that $f(x) \neq x$ and $f(x) \neq -x$ for all $x \in S^{2n}$. We know $\deg(\mathrm{id}) = +1$ on $S^{k}$ and that the antipodal map $a$ has degree $(-1)^{k+1}$. In particular, for $k = 2n$, 
          $$ \deg(a) \;=\; (-1)^{2n+1} \;=\; -1. $$ If $f(x)$ never equals $x$ or $-x$, then for each $x$ we can travel continuously between $x$ and $f(x)$ (and then on to $-x$) without passing through a point where $x$ and $f(x)$ coincide or become antipodes. Then a map $H$ can be constructed, where $H$ would be a homotopy in $S^{2n}$ between $\mathrm{id}$ and $a$. But $\mathrm{id}$ has degree $+1$, while $a$ has degree $-1$. Since degree is a homotopy invariant, no such homotopy can exist. Thus there must be some point $x \in S^{2n}$ such that $f(x)=x$ or $f(x)=-x$.
          \item Let $g: \mathbb{R}P^{2n} \to \mathbb{R}P^{2n}$ be an arbitrary continuous map. We can lift $g$ to a map $f: S^{2n} \to S^{2n}$ such that the following diagram commutes with the projection $\pi: S^{2n} \to \mathbb{R}P^{2n}$:
            $$ \begin{array}{ccc}
          S^{2n} & \xrightarrow{f} & S^{2n} \\
          \downarrow{\pi} & & \downarrow{\pi} \\
          \mathbb{R}P^{2n} & \xrightarrow{g} & \mathbb{R}P^{2n}.
        \end{array} $$
        By Part 1, there is an $x \in S^{2n}$ with $f(x) = x$ or $f(x) = -x$. Project down via $\pi$ to find that $\pi(x) = [x] \quad\text{in}\; \mathbb{R}P^{2n}$ and $g(\pi(x)) \;=\; g([x]) \;=\; \pi(f(x)).$
        \par If $f(x) = x$, then $\pi(f(x)) = [x]$, so $g([x]) = [x]$ is a fixed point. If $f(x) = -x$, then $\pi(-x) = [x]$ again (since $x \sim -x$ in projective space), so $g([x]) = [x]$ still holds. Hence $[x]$ is a fixed point of $g$ in $\mathbb{R}P^{2n}$. Thus every map $\mathbb{R}P^{2n} \to \mathbb{R}P^{2n}$ must have a fixed point.
        \item 
    \end{itemize}
\end{newproof}


\begin{statement}[4]
  Let $f:S^n \to S^n$ be a map of degree zero. Show that there exist points $x,y \in S^n$ with $f(x)=x$ and $f(y)=-y$. Use this to 
  show that if $F$ is a continuous vector field defined on the unit ball $D^n$ in $\mathbb{R}^n$ such that $F(x) \neq 0$ 
  for all $x$, then there exists a point on $\partial D^n$ where $F$ points radially outward and another point on 
  $\partial D^n$ where $F$ points radially inward. 
\end{statement}
\begin{newproof}
    \begin{itemize}
        \item Let $f: S^n \to S^n$ be a map of degree zero. Suppose that $f(x) \neq x$ for every $x \in S^n$. We can try to construct a homotopy $H$ between $f$ and $\mathrm{id}$ by continuously ``sliding'' each point $f(x)$ to $x$:
            $$ H(x,t) \;=\; \frac{(1-t)\,f(x) + t\,x}{\,\|(1-t)\,f(x) + t\,x\|\,}, \quad t \in [0,1]. $$
            Since $f(x) \neq x$ for all $x$, this formula never hits $0$ in $\mathbb{R}^{n+1}$, so $H$ remains on the sphere. Thus $f$ would be homotopic to $\mathrm{id}$, but $\deg(f) = 0 \neq 1 = \deg(\mathrm{id})$, so no such homotopy can exist. Thus there must exists some $x \in S^n$ such that $f(x)=x$.
            \par We can use a similar argument for $y$, because if $f(y) \neq -y$ for all $y \in S^n$, we can attempt the homotopy 
            $$ K(y,t) \;=\; \frac{(1-t)\,f(y) \;+\; t\bigl(-y\bigr)}{\|(1-t)\,f(y) + y)\|}, \quad t \in [0,1], $$
            yielding a homotopy between $f$ and the antipodal map $a(y)=-y$. Since $\deg(a) = (-1)^{n+1} \neq 0$, this too contradicts $\deg(f)=0$. Hence there must be some $y$ satisfying $f(y) = -y$.
        \item Suppose $F$ is a continuous vector field on the closed unit ball $D^n \subset \mathbb{R}^n$. Since $F(x) \neq 0$ for every $x \in D^n$, we may restrict to the boundary $S^{n-1}$ and define:
        $$ f: S^{n-1} \;\longrightarrow\; S^{n-1}, \quad f(x) = \frac{F(x)}{\|F(x)\|}. $$
        Because $\deg(f) = 0$, by Step 1 there exists an $x \in S^{n-1}$ with $f(x) = x$ and a $y \in S^{n-1}$ with $f(y) = -y.$ At $x$, we have $f(x)=x$, meaning 
          $$
            \frac{F(x)}{\|F(x)\|} = x,
          $$
          so $F(x)$ is a positive scalar multiple of $x$. Thus $F(x)$ is pointing directly outward. At $y$, we have $f(y) = -y$, meaning 
          $$
            \frac{F(y)}{\|F(y)\|} = -y,
          $$
          so $F(y)$ is a negative scalar multiple of $y$. Thus $F(y)$ is pointing inward.
    \end{itemize}
\end{newproof}

\begin{statement}[5]
  For an invertible linear transformation $f: \R^n \to \R^n$ show that the induced map on $H_n$ 
  $(\R^n, \R^n - \{0\}) \cong \tilde{H}_{n-1}(\R^n - \{0\}) \cong \Z$ is $\mathbbm{1}$ or $-\mathbbm{1}$ 
  according to whether the determinant of $f$ is positive or negative.
\end{statement}
\begin{newproof}

\end{newproof}

\begin{statement}[6]
  A polynomial $f(z)$ with complex coefficients, viewed as a map $\mathbb{C} \to \mathbb{C}$, can always be extended 
  to a continuous map of one-point compactifications $\hat{f}:S^2 \to S^2$. Show that the degree of $\hat{f}$ 
  equals the degree of $f$ as a polynomial. Show also that the local degree of $\hat{f}$ at a root of $f$ 
  is the multiplicity of the root.
\end{statement}
\begin{newproof}

\end{newproof}

\end{document}
