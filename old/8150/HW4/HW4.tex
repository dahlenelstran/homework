\documentclass[12pt]{article}
 
\usepackage[margin=.5in]{geometry} 
\usepackage{amsmath,amsthm,amssymb,outlines}
\usepackage{graphicx,tikzsymbols,tcolorbox}
\renewcommand\qedsymbol{$\blacksquare$}
\newcommand{\Z}{\mathbb{Z}}
\newcommand{\R}{\mathbb{R}}
\tcbuselibrary{most}

\newtcolorbox{newtitle}{
  enhanced,
  colframe=black,
  colback=white,
  boxsep=5pt,
  arc=8pt,
  sharp corners=south,
  borderline={0.5pt}{0pt}{black},
  borderline={1.8pt}{-5pt}{black},
  after skip=30pt
}

\newtcolorbox[auto counter]{statement}[1][]{
  enhanced,
  breakable,
  title={Exercise \ifx\\#1\\\thetcbcounter\else#1\fi},
  colframe=black,
  colback=white,
  colbacktitle=white,
  fonttitle=\bfseries,
  coltitle=black,
  attach boxed title to top left={yshift=-0.25mm-\tcboxedtitleheight/2,yshifttext=2mm-\tcboxedtitleheight/2, xshift=2mm},
  boxed title style={boxrule=0.5mm}
}

\newtcolorbox{newproof}{
  enhanced,
  breakable,
  frame hidden,
  colback=white,
  title={Proof.},
  fonttitle=\bfseries,
  coltitle=black,
  colbacktitle=white,
  boxed title style={boxrule=0.5mm},
  attach boxed title to top left={yshift=-0.25mm-\tcboxedtitleheight/2,yshifttext=2mm-\tcboxedtitleheight/2, xshift=2mm},
  borderline west={1.5pt}{8pt}{black},
  after upper={\hfill $\blacksquare$}
}

\begin{document}

\begin{newtitle}
  \begin{center}
    \textbf{\Huge 8150 Homework 4}
  \end{center}
  \textbf{Dahlen Elstran} \hfill \textbf{\today}
\end{newtitle}

\section{Stein Problems}

\begin{statement}[3.9.2]
  Let $u$ be a harmonic function in the unit disc that is continuous on its closure. Deduce Poisson's integral formula 
  $$ u(z_0)=\frac{1}{2\pi}\int^{2\pi}_0 \frac{1 - \vert z_0 \vert ^2}{\vert e^{i \theta} - z_0 \vert ^2} u(e^{i \theta}) d \theta \text{ for } \vert z_0 \vert < 1 $$
  from the special case $z_0=0$ (the mean value theorem). Show that if $z_0=re^{i \phi}$, then 
  $$ \frac{1 - \vert z_0 \vert ^2}{\vert e^{i \theta} - z_0 \vert ^2} = \frac{1 - r^2}{1-2r \cos(\theta - \phi) + r^2} = P_r(\theta - \phi) $$
  and we recover the expression for the Poisson kernel derived in the exercises of the previous chapter.
  \par (Hint: Set $u_0(z)=u(T(z))$ where 
  $$ T(z) = \frac{z_0 - z}{1 - \bar{z_0}z}. $$ 
  Prove that $u_0$ is harmonic. Then apply the mean value theorem to $u_0$, and make a change of variables in the integral.)
\end{statement}
\begin{newproof}

\end{newproof}

\begin{statement}[3.9.3]
  If $f(z)$ is holomorphic in the deleted neighborhood $\{ 0 < \vert z - z_0 \vert r \}$ and has a pole of order $k$ at $z_0$, then
  we can write 
  $$ f(z)=\frac{a_{-k}}{(z-z_0)^k} + \dots + \frac{a_{-1}}{(z-z_0)} + g(z) $$
  where $g$ is holomorphic in the disc $\{ \vert z - z_0 \vert < r\}$. There is a generalization of this expansion that holds 
  even if $z_0$ is an essential singularity. This is a special case of the \textbf{Laurent series expansion}, which is valid in an 
  even more general settings. 
  \par Let $f$ be holomorphic in a region containing the annulus $\{ z : r_1 \leq \vert z - z_0 \vert \leq r_2 \}$ where 
  $0 < r_1 < r_2$. Then,
  $$ f(z) = \sum^{\infty}_{n=-\infty} a_n(z - z_0)^n $$ 
  where the series converges absolutely in the interior of the annulus. To prove this, it suffices to write 
  $$ f(z) = \frac{1}{2\pi i} \int_{C_{r_2}} \frac{f(\xi)}{\xi - z}d \xi - \frac{1}{2 \pi i} \int_{C_{r_1}} \frac{f(\xi)}{\xi - z} d \xi $$
  when $r_1 < \vert z - z_0 \vert < r_2$, and argue as in the proof of Theorem 4.4, Chapter 2. Here $C_{r_1}$ and 
  $C_{r_2}$ are the circles bounding the annulus.
\end{statement}
\begin{newproof}

\end{newproof}

\begin{statement}[4.4.1]

\end{statement}
\begin{newproof}

\end{newproof}

\begin{statement}[4.4.2]

\end{statement}
\begin{newproof}

\end{newproof}

\begin{statement}[4.4.3]

\end{statement}
\begin{newproof}

\end{newproof}

\begin{statement}[4.4.6]

\end{statement}
\begin{newproof}

\end{newproof}

\begin{statement}[4.4.7]

\end{statement}
\begin{newproof}

\end{newproof}

\begin{statement}[4.4.8]

\end{statement}
\begin{newproof}

\end{newproof}

\section{Tie Problems}

\begin{statement}[1]

\end{statement}
\begin{newproof}

\end{newproof}

\begin{statement}[2]

\end{statement}
\begin{newproof}

\end{newproof}

\begin{statement}[3]

\end{statement}
\begin{newproof}

\end{newproof}

\begin{statement}[4]

\end{statement}
\begin{newproof}

\end{newproof}

\end{document}
