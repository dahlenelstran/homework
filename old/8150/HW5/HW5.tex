\documentclass[12pt]{article}
 
\usepackage[margin=.5in]{geometry} 
\usepackage{amsmath,amsthm,amssymb,outlines}
\usepackage{graphicx,tikzsymbols,tcolorbox}
\renewcommand\qedsymbol{$\blacksquare$}
\newcommand{\Z}{\mathbb{Z}}
\newcommand{\R}{\mathbb{R}}
\newcommand{\C}{\mathbb{C}}
\tcbuselibrary{most}

\newtcolorbox{newtitle}{
  enhanced,
  colframe=black,
  colback=white,
  boxsep=5pt,
  arc=8pt,
  sharp corners=south,
  borderline={0.5pt}{0pt}{black},
  borderline={1.8pt}{-5pt}{black},
  after skip=30pt
}

\newtcolorbox[auto counter]{statement}[1][]{
  enhanced,
  breakable,
  title={Exercise \ifx\\#1\\\thetcbcounter\else#1\fi},
  colframe=black,
  colback=white,
  colbacktitle=white,
  fonttitle=\bfseries,
  coltitle=black,
  attach boxed title to top left={yshift=-0.25mm-\tcboxedtitleheight/2,yshifttext=2mm-\tcboxedtitleheight/2, xshift=2mm},
  boxed title style={boxrule=0.5mm}
}

\newtcolorbox{newproof}{
  enhanced,
  breakable,
  frame hidden,
  colback=white,
  title={Proof.},
  fonttitle=\bfseries,
  coltitle=black,
  colbacktitle=white,
  boxed title style={boxrule=0.5mm},
  attach boxed title to top left={yshift=-0.25mm-\tcboxedtitleheight/2,yshifttext=2mm-\tcboxedtitleheight/2, xshift=2mm},
  borderline west={1.5pt}{8pt}{black},
  after upper={\hfill $\blacksquare$}
}

\begin{document}

\begin{newtitle}
  \begin{center}
    \textbf{\Huge Homework 5}
  \end{center}
  \textbf{Dahlen Elstran} \hfill \textbf{\today}
\end{newtitle}

\section{Stein Problems}

\begin{statement}[8.5.1]
  A holomorphic mapping $f: U \to V$ is a \textbf{local bijection} on $U$ if for every $z \in U$ there exists an 
  open disc $D \subset U$ centered at $z$, so that $f: D \to f(D)$ is a bijection. Prove that a holomorphic 
  map $f:U \to V$ is a local bijection on $U$ if and only if $f'(z) \neq 0$ for all $z \in U$. 
\end{statement}
\begin{newproof}
    ($\Rightarrow$) Fix $z_0\in U$. Assume for contradiction that $f'(z_0)=0$. Write the Taylor expansion at $z_0$:
        $$ f(z)=f(z_0)+a_k\,(z-z_0)^k+\cdots , \qquad a_k\neq0,\; k\ge2 $$
    Choose $\rho>0$ so small that the closed disc $\overline{D_\rho}(z_0)\subset U$ and $|f(z)-f(z_0)|<\tfrac12|a_k|\rho^k$ for $|z-z_0|\le\rho$. For $\zeta=e^{2\pi i/k}$ set $z_1=z_0+\rho$ and $z_2=z_0+\rho\zeta$. Then $|z_1-z_0|=|z_2-z_0|=\rho$ but $z_1\neq z_2$ and
        $$ f(z_j)=f(z_0)+a_k\rho^k+R_j,\qquad |R_j|<\frac12|a_k|\rho^k , $$
    so $|f(z_j)-f(z_0)-a_k\rho^k|<\frac12|a_k|\rho^k$ for $j=1,2$. By the triangle inequality $f(z_1)=f(z_2)$. Hence $f$ is not injective on any disc about $z_0$, contradicting the hypothesis that $f$ is a local bijection. Therefore $f'(z)\neq0$ for all $z\in U$.
    \smallskip
    \par ($\Leftarrow$) Let $z_0\in U$ with $f'(z_0)\neq0$. Then there exists $r>0$ such that the restricted map
        $$ f:\; D_r(z_0)\;\longrightarrow\; f(D_r(z_0)) $$ 
    is biholomorphic, and it's bijective onto an open disc $f(D_r(z_0))\subset V$. Thus $f$ is a local bijection on $U$.
\end{newproof}


\begin{statement}[8.5.2]
  Suppose $F(z)$ is holomorphic near $z=z_0$ and $F(z_0)=F'(z_0)=0$, while $F''(z_0) \neq 0$. Show that 
  there are two curves $\Gamma_1$ and $\Gamma_2$ that pass through $z_0$, are orthogonal at 
  $z_0$, and so that $F$ restricted to $\Gamma_1$ is real and has a minimum at $z_0$, while $F$ 
  restricted to $\Gamma_2$ is also real but has a maximum at $z_0$. 
\end{statement}
\begin{newproof}
  Let $F$ be holomorphic near $z_0$ with $ F(z_0)=F'(z_0)=0, F''(z_0)\neq0. $ Write the second–order Taylor expansion
  $$F(z_0+w)=\frac{F''(z_0)}{2}\,w^{\,2}+o(|w|^{2}), \qquad w:=z-z_0.$$
  Denote $a:=\tfrac12F''(z_0)\neq0$ and let $\varphi\in[0,\pi)$ satisfy 
  $ e^{-2i\varphi}a\in\R^+. $  
  Then the rays  
  $$\gamma_1:\;w=t\,e^{\,i\varphi},\qquad 
    \gamma_2:\;w=t\,e^{\,i(\varphi+\pi/2)},\qquad t\in\R,$$ 
  form an orthogonal cross at $w=0$.  Along these directions
  $$
      F(z_0+t\,e^{i\varphi})
      =a\,t^{2}+o(t^{2})\in\R,\quad
      F(z_0+t\,e^{i(\varphi+\pi/2)})
      =-\,a\,t^{2}+o(t^{2})\in\R.
  $$
  Define the real–analytic functions  
  $$G_1(z):=\text{Im}(e^{-2i\varphi}F(z)\bigr),\qquad 
    G_2(z):=\text{Im}(e^{-2i(\varphi+\pi/2)}F(z)\bigr).$$  
  Because $G_j(z_0)=0$ and $\nabla G_j(z_0)\neq0$, each level set $G_j^{-1}(0)$ is, by the implicit function theorem, a real $1$-dimensional smooth curve through $z_0$ whose tangent direction is that of $\gamma_j$.  
  Call these curves $\Gamma_1$ and $\Gamma_2$; they meet orthogonally at $z_0$.

  \smallskip
  Along $\Gamma_1$ we have $e^{-2i\varphi}F\in\R$, hence $F$ is real-valued; Taylor’s formula gives  
  $$F(z_0+t\,e^{i\varphi})=a\,t^{2}+o(t^{2})\quad (t\to0),$$
  which is $\ge0$ for $t$ small, with equality only at $t=0$.  Thus $F|_{\Gamma_1}$ attains a minimum at $z_0$.

  Similarly, along $\Gamma_2$ we obtain  
  $$F(z_0+t\,e^{i(\varphi+\pi/2)})=-\,a\,t^{2}+o(t^{2}),$$
  which is $\le0$ near $t=0$; hence $F|_{\Gamma_2}$ is real-valued and achieves a maximum at $z_0$.
\end{newproof}

\begin{statement}[8.5.8]
Find a harmonic function $u$ in the open first quadrant that extends continuously up to the boundary except at the points $0$ and $1$, and that takes on the following boundary values: $u(x,y) = 1$ on the half-lines $\{ y = 0,\ x > 1 \}$ and $\{ x = 0,\ y > 0 \}$, and $u(x,y) = 0$ on the segment $\{ 0 < x < 1,\ y = 0 \}$.

\par \textbf{[Hint:]} Find conformal maps $F_1, F_2, \dots, F_5$ indicated in Figure 11. Note that
$$
\frac{1}{\pi} \arg(z)
$$
is harmonic on the upper half-plane, equals $0$ on the positive real axis, and $1$ on the negative real axis.
\end{statement}
\begin{newproof}
  Let $ z=x+iy,\qquad Q=\{x>0,\;y>0\}\setminus\{0,1\}. $ Set $ w=F_1(z)=z^{2}. $ Because $\arg z\in(0,\pi/2)$ in $Q$, we have $\arg w\in(0,\pi)$, so $F_1$ sends $Q$ conformally onto $ H=\{\,\text{Im}\,w>0\}. $ Then
  $$ \begin{array}{ccc}
        z\in(0,1)&\longmapsto&w\in(0,1),\\[6pt]
        z\in(1,\infty)&\longmapsto&w\in(1,\infty),\\[6pt]
        z=iy,\;y>0&\longmapsto&w=-y^{2}\in(-\infty,0).
    \end{array} $$
    Thus the three boundary pieces become, respectively, $(0,1)$, $(1,\infty)$, and $(-\infty,0)$ on the real axis of $H$.
    Then use the map $ \zeta=F_2(w)=\frac{w}{1-w}, w\in H $, so that if $w\in(0,1)$ then $\zeta>0$; if $w>1$ or $w<0$ then $\zeta<0$. Because $F_2$ is real-analytic and preserves $H$, it is conformal on $H$.
    \par A harmonic function on $H$ with the required boundary values is
        $$ U(\zeta)=\frac{1}{\pi}\,\text{Arg}\,\zeta, \qquad 0<\text{Arg}\,\zeta<\pi .$$
    Clearly, $U=0$ on the positive real axis and $U=1$ on the negative real axis. Define
    $$ u(z)=U\!\bigl(F_2(F_1(z))\bigr)
      =\frac{1}{\pi}\,\text{Arg}\!\left(\frac{z^{2}}{1-z^{2}}\right),
      \qquad z\in Q. $$
    Because $F_1$ and $F_2$ are conformal, $u$ is harmonic in $Q$ and extends continuously (except at $z=0,1$) to the boundary. Then we have
    \begin{align*}
        z\in(0,1) &: \dfrac{z^{2}}{1-z^{2}}>0, \text{ so } \text{Arg}=0 \text{ and } u=0. \\
        z\in(1,\infty) &: \dfrac{z^{2}}{1-z^{2}}<0, \text{ so } \text{Arg}=\pi \text{ and } u=1.\\
        z=iy, y>0 &: \dfrac{z^{2}}{1-z^{2}}<0, \text{ so } \text{Arg}=\pi \text{ and } u=1.
    \end{align*}
    Hence $u$ satisfies all the required boundary values:
    $$ u=1 \text{ on } \{y=0,\;x>1\}\ \text{ and }\ \{x=0,\;y>0\},\qquad u=0 \text{ on } \{0<x<1,\;y=0\}. $$
    Thus the function  
    $$ u(x,y)=\frac{1}{\pi}\,\text{Arg}\!\left(\frac{(x+iy)^{2}}{1-(x+iy)^{2}}\right) $$
    works.
\end{newproof}

\begin{statement}[8.5.9]
    Prove that the function $u$ defined by
        $$ u(x, y) = \operatorname{Re} \left( \frac{i + z}{i - z} \right) \quad \text{and} \quad u(0, 1) = 0 $$
    is harmonic in the unit disc and vanishes on its boundary. Note that $u$ is not bounded in $\mathbb{D}$.
\end{statement}
\begin{newproof}
    Let $z = x + iy$ and define
    $$ u(x, y) = \text{Re} \left( \frac{i + z}{i - z} \right), \quad u(0,1) = 0. $$
    First, we show that $u$ is harmonic in the unit disc. Let 
    $$ f(z) = \frac{i + z}{i - z}. $$
    This is holomorphic on all of $\mathbb{D}$, since $i \notin \overline{\mathbb{D}}$. The real part of a holomorphic function is harmonic wherever the function is holomorphic, so $u = \text{Re}(f(z))$ is harmonic on $\mathbb{D}$.
    \par To show that $u$ vanishes on $\partial \mathbb{D}$, let $z = e^{i\theta}$ with $\theta \in [0, 2\pi)$. Then
    $$ f(z) = \frac{i + e^{i\theta}}{i - e^{i\theta}}. $$
    We claim that $f(z)$ is purely imaginary for $|z| = 1$. Observe that $f$ maps $\mathbb{D}$ conformally onto the right half-plane, and thus maps $\partial \mathbb{D}$ to the boundary of that half-plane, which is the imaginary axis. So $\text{Re}(f(z)) = 0$ for all $|z| = 1$, and hence $u(x,y) = 0$ on $\partial \mathbb{D}$.
\end{newproof}


\begin{statement}[8.5.10]
    Let $F : \mathbb{H} \to \mathbb{C}$ be a holomorphic function that satisfies
        $$ |F(z)| \leq 1 \quad \text{and} \quad F(i) = 0. $$
    \par Prove that
        $$ |F(z)| \leq \left| \frac{z - i}{z + i} \right| \quad \text{for all } z \in \mathbb{H}. $$
\end{statement}
\begin{newproof}
    Define
    $$ \phi(z) = \frac{z - i}{z + i}. $$
    This maps the upper half-plane $\mathbb{H}$ conformally onto the unit disc $\mathbb{D} = \{ w \in \mathbb{C} : |w| < 1 \}$.
    Clearly,
    \begin{itemize}
        \item $\phi$ is holomorphic on $\mathbb{H}$
        \item For all $z \in \mathbb{H}$, $\text{Im}(z) > 0$, so $z + i \neq 0$ and $\phi$ is well-defined
        \item The image lies in $\mathbb{D}$ because 
            $$ |\phi(z)| = \left| \frac{z - i}{z + i} \right| < 1 \quad \text{for all } z \in \mathbb{H}. $$
    \end{itemize}
    Define the composition
    $$ G(w) = F\bigl(\phi^{-1}(w)\bigr). $$
    Then $G : \mathbb{D} \to \mathbb{C}$ is holomorphic, because $\phi^{-1}$ is holomorphic and maps $\mathbb{D}$ to $\mathbb{H}$. Since $|F(z)| \leq 1$ on $\mathbb{H}$, we have
    $$ |G(w)| = |F(\phi^{-1}(w))| \leq 1 \quad \text{for all } w \in \mathbb{D}. $$
    Also, note that $\phi(i) = 0$, so
    $$ G(0) = F(\phi^{-1}(0)) = F(i) = 0. $$
    Therefore, $G$ is a holomorphic function on $\mathbb{D}$, bounded by $1$ in modulus, and vanishing at $0$. By the Schwarz Lemma,
    $$ |G(w)| \leq |w| \quad \text{for all } w \in \mathbb{D}. $$
    Apply this inequality to $w = \phi(z)$:
    $$ |F(z)| = |G(\phi(z))| \leq |\phi(z)| = \left| \frac{z - i}{z + i} \right|. $$
    Hence,
    $$ |F(z)| \leq \left| \frac{z - i}{z + i} \right| \quad \text{for all } z \in \mathbb{H}. $$
\end{newproof}

\begin{statement}[8.5.11]
    Show that if $f : D(0, R) \to \mathbb{C}$ is holomorphic, with $|f(z)| \leq M$ for some $M > 0$, then
        $$ \left| \frac{f(z) - f(0)}{M^2 - \overline{f(0)} f(z)} \right| \leq \frac{|z|}{MR}. $$
    \par \textbf{[Hint:]} Use the Schwarz lemma.
\end{statement}
\begin{newproof}
    Define a new function
    $$ F(z) = \frac{f(z)}{M}, $$
    so that $|F(z)| \leq 1$ on $D(0, R)$ and $F$ is holomorphic.
    Let $w = F(z)$ and $w_0 = F(0) = a/M$. Since $|F(z)| \leq 1$, we can define a transformation that sends $w_0$ to $0$:
    $$ \phi(w) = \frac{w - w_0}{1 - \overline{w_0} w}. $$
    This map sends the unit disc to itself, is holomorphic, and satisfies $\phi(w_0) = 0$. Now define the composition
    $$ g(z) = \phi(F(z)) = \frac{F(z) - F(0)}{1 - \overline{F(0)} F(z)}. $$
    Then $g$ is holomorphic on $D(0, R)$, maps into the unit disc, and satisfies $g(0) = 0$. Define
    $$ h(z) = g(Rz), $$
    which is holomorphic on $D(0, 1)$ with $h(0) = 0$ and $|h(z)| < 1$ for all $|z| < 1$. By the Schwarz Lemma,
    $$ |h(z)| \leq |z| \quad \Rightarrow \quad |g(w)| \leq \frac{|w|}{R}. $$
    So for all $z \in D(0, R)$,
    $$ \left| \frac{F(z) - F(0)}{1 - \overline{F(0)} F(z)} \right| \leq \frac{|z|}{R}. $$
    We know $F(z) = f(z)/M$, so that
    $$ \frac{F(z) - F(0)}{1 - \overline{F(0)} F(z)} = \frac{f(z)/M - a/M}{1 - \overline{a}/M \cdot f(z)/M} = \frac{f(z) - a}{M^2 - \overline{a} f(z)}. $$
    Therefore,
    $$ \left| \frac{f(z) - f(0)}{M^2 - \overline{f(0)} f(z)} \right| \leq \frac{|z|}{MR}. $$
\end{newproof}

\newpage

\begin{statement}[8.5.12]
    A complex number $w \in \mathbb{D}$ is a \textbf{fixed point} for the map $f : \mathbb{D} \to \mathbb{D}$ if $f(w) = w$.
    \begin{itemize}
        \item[(a)] Prove that if $f : \mathbb{D} \to \mathbb{D}$ is analytic and has two distinct fixed points, then $f$ is the identity, that is, $f(z) = z$ for all $z \in \mathbb{D}$.
    
        \item[(b)] Must every holomorphic function $f : \mathbb{D} \to \mathbb{D}$ have a fixed point? \textbf{[Hint:]} Consider the upper half-plane.
    \end{itemize}
\end{statement}
\begin{newproof}
    Let $f : \mathbb{D} \to \mathbb{D}$ be holomorphic.
    (a) Suppose $f$ has two distinct fixed points $z_1, z_2 \in \mathbb{D}$ with $z_1 \neq z_2$, so that
    $$ f(z_1) = z_1, \quad f(z_2) = z_2. $$
    Fix $z_0 \in \mathbb{D}$. Define an automorphism of $\mathbb{D}$:
    $$ \phi(z) = \frac{z - z_1}{1 - \overline{z_1} z}, \quad \text{so that } \phi(z_1) = 0. $$
    Then define the conjugated function
    $$ F(z) = \phi \circ f \circ \phi^{-1}(z). $$
    Since $\phi$ is an automorphism of $\mathbb{D}$ and $f : \mathbb{D} \to \mathbb{D}$ is holomorphic, the function $F : \mathbb{D} \to \mathbb{D}$ is holomorphic.
    Now, since $f(z_1) = z_1$, we have $F(0) = 0$. Let $w = \phi(z_2) \neq 0$, since $z_2 \neq z_1$. Then
    $$ F(w) = \phi(f(\phi^{-1}(w))) = \phi(\phi^{-1}(w)) = w, $$
    so $F$ has two distinct fixed points: $0$ and $w \neq 0$.
    By the Schwarz lemma, if a holomorphic map $g : \mathbb{D} \to \mathbb{D}$ fixes $0$ and is not the identity, then
    $$ |g(z)| < |z| \quad \text{for all } z \neq 0. $$
    But $F$ fixes $0$ and $w \neq 0$, so this strict inequality fails. Hence $F(z) = z$ for all $z \in \mathbb{D}$. Therefore,
    $$ f = \phi^{-1} \circ F \circ \phi = \phi^{-1} \circ \phi = \text{id}, $$
    so $f(z) = z$ for all $z \in \mathbb{D}$.

    (b) Not every holomorphic function $f : \mathbb{D} \to \mathbb{D}$ has a fixed point.
    Consider the map
    $$ f(z) = \frac{z + a}{1 + \overline{a}z}, \quad a \in \mathbb{D} \setminus \{0\}. $$
    This is an automorphism of $\mathbb{D}$ that maps $0$ to $a$. Then
    $$ f(0) = a \neq 0, \quad f(a) = \frac{2a}{1 + |a|^2} \neq a, $$
    and unless $a = 0$, this function has no fixed point in $\mathbb{D}$.
\end{newproof}


\begin{statement}[8.5.13]
\end{statement}

\begin{statement}[8.5.14]
    Prove that all conformal mappings from the upper half-plane $\mathbb{H}$ to the unit disc $\mathbb{D}$ take the form
    $$ e^{i\theta} \frac{z - \beta}{z - \overline{\beta}}, \quad \theta \in \mathbb{R} \text{ and } \beta \in \mathbb{H}. $$
\end{statement}
\begin{newproof}
    We know that all conformal maps between simply connected domains (other than $\mathbb{C}$) are given by the Riemann Mapping Theorem, and all automorphisms of the unit disc $\mathbb{D}$ are of the form
    $$ \phi_a(w) = e^{i\theta} \cdot \frac{w - a}{1 - \overline{a}w}, \quad |a| < 1,\, \theta \in \mathbb{R}. $$
    Also, the Möbius map 
    $$ T(z) = \frac{z - i}{z + i} $$
    is a conformal bijection from $\mathbb{H}$ to $\mathbb{D}$. Any conformal map $f : \mathbb{H} \to \mathbb{D}$ can be written as a composition
    $$ f(z) = \phi(T(z)), $$
    where $T(z) = \frac{z - i}{z + i}$ and $\phi : \mathbb{D} \to \mathbb{D}$ is a disc automorphism. So,
    $$ f(z) = e^{i\theta} \cdot \frac{T(z) - a}{1 - \overline{a} T(z)}, \quad \text{for some } a \in \mathbb{D},\; \theta \in \mathbb{R}. $$
    Now substitute the explicit expression for $T(z)$:
    $$ f(z) = e^{i\theta} \cdot \frac{\dfrac{z - i}{z + i} - a}{1 - \overline{a} \cdot \dfrac{z - i}{z + i}}. $$
    Multiply numerator and denominator by $z + i$: 
    $$ f(z) = e^{i\theta} \cdot \frac{(z - i) - a(z + i)}{(z + i) - \overline{a}(z - i)}. $$
    Simplify numerator:
    $$ (z - i) - a(z + i) = z(1 - a) - i(1 + a), $$
    Simplify denominator:
    $$ (z + i) - \overline{a}(z - i) = z(1 - \overline{a}) + i(1 + \overline{a}). $$
    So the full expression is a Möbius transformation of the form 
    $$ f(z) = e^{i\theta} \cdot \frac{A z + B}{C z + D}. $$
    Now, since $f$ maps $\mathbb{H}$ to $\mathbb{D}$ conformally, it must be a Möbius transformation that maps the upper half-plane to the disc. A standard form of such a map is:
    $$ f(z) = e^{i\theta} \cdot \frac{z - \beta}{z - \overline{\beta}}, \quad \beta \in \mathbb{H}. $$
    For $\beta \in \mathbb{H}$, note that $f(\beta) = 0$, $f(\overline{\beta}) = \infty$, $|f(z)| < 1$ for all $z \in \mathbb{H}$, and $f$ is a Möbius transformation that maps $\mathbb{H}$ onto $\mathbb{D}$.
    Therefore, any conformal map $f : \mathbb{H} \to \mathbb{D}$ has the form 
    $$ f(z) = e^{i\theta} \cdot \frac{z - \beta}{z - \overline{\beta}}, \quad \beta \in \mathbb{H},\ \theta \in \mathbb{R}.$$
\end{newproof}

\begin{statement}[8.5.15]
    Here are two properties enjoyed by automorphisms of the upper half-plane.
    \begin{itemize}
        \item[(a)] Suppose $\Phi$ is an automorphism of $\mathbb{H}$ that fixes three distinct points on the real axis. Then $\Phi$ is the identity.

        \item[(b)] Suppose $(x_1, x_2, x_3)$ and $(y_1, y_2, y_3)$ are two pairs of three distinct points on the real axis with
        $$ x_1 < x_2 < x_3 \quad \text{and} \quad y_1 < y_2 < y_3. $$
        Prove that there exists (a unique) automorphism $\Phi$ of $\mathbb{H}$ so that $\Phi(x_j) = y_j$, $j = 1,2,3$. The same conclusion holds if $y_3 < y_1 < y_2$ or $y_2 < y_3 < y_1$.
    \end{itemize}
\end{statement}
\begin{newproof}
    (a) We know that automorphisms of $\mathbb{H}$ are of the form
    $$ \Phi(z) = \frac{az + b}{cz + d}, $$
    where $a,b,c,d \in \mathbb{R}$ and $ad - bc > 0$. Suppose $\Phi$ fixes $x_1,x_2,x_3 \in \mathbb{R}$ with $x_1,x_2,x_3$ distinct. Since $\Phi(x_j) = x_j$ for $j = 1,2,3$, and Möbius transformations are determined by their values on three distinct points, it follows that $\Phi(z) = z$ for all $z \in \mathbb{H}$. Thus, $\Phi$ is the identity.

    (b) Now, suppose $(x_1, x_2, x_3)$ and $(y_1, y_2, y_3)$ are two ordered triples of distinct real numbers satisfying
    $$ x_1 < x_2 < x_3 \quad \text{and} \quad y_1 < y_2 < y_3, $$
    To prove one exists, note that automorphisms of $\mathbb{H}$ are precisely those Möbius transformations with real coefficients and positive determinant, and they preserve $\mathbb{H}$. Thus, there exists a Möbius transformation $\Phi$ with real coefficients, mapping $x_j \mapsto y_j$ for $j=1,2,3$. To prove uniqueness, suppose there were two automorphisms $\Phi_1$ and $\Phi_2$ satisfying $\Phi_1(x_j) = \Phi_2(x_j) = y_j$ for $j=1,2,3$. Then $\Phi_2^{-1} \circ \Phi_1$ would be an automorphism of $\mathbb{H}$ fixing three distinct points, hence the identity by part (a). Thus $\Phi_1 = \Phi_2$, and the automorphism is unique.
\end{newproof}

\begin{statement}[8.5.16]
\end{statement}

\begin{statement}[8.5.17]
\end{statement}

\section{Additional Problems}

\begin{statement}[1]
    Let $\Omega = \{ z : |z - 1| < \sqrt{2},\ |z + 1| < \sqrt{2} \}$. Find a bijective conformal map from $\Omega$ to the upper half-plane $\mathbb{H}$.
\end{statement}
\begin{newproof}
    Note that $\Omega$ is the intersection of two open disks of radius $\sqrt{2}$ centered at $1$ and $-1$, respectively. Let us define
    $$ f(z) = \frac{z - i}{z + i}. $$
    This transformation maps:
    \begin{itemize}
        \item $z = i$ to $0$,
        \item $z = -i$ to $\infty$,
        \item the unit circle to the real line,
        \item the upper half of the unit circle (where $\text{Im}(z) > 0$) to the upper half-plane $\mathbb{H}$.
    \end{itemize}
    We claim that $f$ maps $\Omega$ onto $\mathbb{H}$. To justify this, note that the boundaries of $\Omega$ are arcs of the circles $|z - 1| = \sqrt{2}$ and $|z + 1| = \sqrt{2}$. These two circles intersect orthogonally at $z = \pm i$, and the map $f(z) = \frac{z - i}{z + i}$ maps any pair of circles intersecting orthogonally at $z = \pm i$ to rays meeting at the origin in the complex plane. In particular, it maps the circular arcs bounding $\Omega$ to intervals on the real line, and the domain between them to the upper half-plane. Therefore, $f$ maps $\Omega$ conformally and bijectively onto $\mathbb{H}$.
\end{newproof}

\begin{statement}[2]
    Find the fractional linear transformation that maps the circle $|z| = 2$ into $|z + 1| = 1$, the point $-2$ into the origin, and the origin into $i$.
\end{statement}
\begin{newproof}
    A fractional linear transformation is uniquely determined by the images of three distinct points. Define $z_1 = -2$, $z_2 = 0$, $z_3 = 2$ (three points on $|z| = 2$), and let their images be $w_1 = 0$, $w_2 = i$, and $w_3 = \infty$ (since $z_3 = 2$ lies on $|z| = 2$, it is reasonable to map it to a point at infinity to map the circle to a line or another circle). Then the desired transformation is the unique function satisfying:
    $$ f(-2) = 0,\quad f(0) = i,\quad f(2) = \infty. $$
    \par Assume
    $$ f(z) = \lambda \cdot \frac{z + 2}{z - 2} $$
    so that $f(-2) = \lambda \cdot \frac{-2 + 2}{-2 - 2} = 0$ and $f(2) = \lambda \cdot \frac{4}{0} = \infty$.
    Now choose $\lambda$ to satisfy $f(0) = i$:
    $$ f(0) = \lambda \cdot \frac{0 + 2}{0 - 2} = \lambda \cdot \left( \frac{2}{-2} \right) = -\lambda. $$
    So for $f(0) = i$, we need $-\lambda = i$, or $\lambda = -i$. Therefore, the desired transformation is
    $$ f(z) = -i \cdot \frac{z + 2}{z - 2}. $$
    To check that $|z| = 2$ maps to $|z + 1| = 1$, let $z$ be on the circle $|z| = 2$. Then write $z = 2e^{i\theta}$, and we can find
    $$ f(z) = -i \cdot \frac{2e^{i\theta} + 2}{2e^{i\theta} - 2} = -i \cdot \frac{2(e^{i\theta} + 1)}{2(e^{i\theta} - 1)} = -i \cdot \frac{e^{i\theta} + 1}{e^{i\theta} - 1}. $$
    So it sends the unit circle (in this case, $|z| = 2$ scaled) onto a circle centered at $-1$ of radius $1$, i.e., the circle $|w + 1| = 1$.
    Therefore, the final answer is: 
    $$ f(z) = -i \cdot \frac{z + 2}{z - 2}. $$
\end{newproof}

\begin{statement}[3]
    Let $\Omega = \mathbb{D} \setminus (-1, -1/2]$. Find a bijective conformal map from $\Omega$ to the unit disk $\mathbb{D}$. How do you find the most general form of all such maps (you don’t have to explicitly describe the general form, just explain the strategy for obtaining it)?
\end{statement}
\begin{newproof}
    Let
    $$ \phi(z) = i \cdot \frac{1 + z}{1 - z}, $$
    which maps $\mathbb{D}$ conformally onto the upper half-plane $\mathbb{H} = \{ \text{Im}(w) > 0 \}$. Then $\phi(\Omega)$ is the upper half-plane with the slit $(\phi(-1), \phi(-1/2)]$ removed. We find that
    $$ \phi(-1) = i \cdot \frac{1 - 1}{1 + 1} = 0, \quad \phi(-1/2) = i \cdot \frac{1 - 1/2}{1 + 1/2} = i \cdot \frac{1/2}{3/2} = \frac{i}{3}. $$
    So $\phi(\Omega) = \mathbb{H} \setminus \left[0, \frac{i}{3}\right]$, a vertical slit segment from $0$ to $i/3$ removed from $\mathbb{H}$.
    \par We now look for a conformal map $\psi : \mathbb{H} \setminus [0, i/3] \to \mathbb{H}$. We know that $\Omega$ can be mapped conformally onto the upper half-plane using the square root function:
    $$ \psi(w) = \sqrt{w - c}, \quad \text{where } c \text{ is the endpoint of the slit}. $$
    In this case, the slit ends at $i/3$, so we instead first rotate the slit to lie on the real axis. Define
    $$ T(w) = -iw. $$
    Then $T$ rotates the vertical segment $[0, i/3]$ to the real segment $[0, 1/3]$. Now define
    $$ \psi(w) = \sqrt{w}. $$
    So the composition $\sqrt{-iw}$ maps $\mathbb{H} \setminus [0, i/3]$ onto a quadrant (or half-plane). Further composing with another Möbius map sends the result back to $\mathbb{D}$.
    \par Let
    $$ f(z) = \chi \circ \psi \circ T \circ \phi(z), $$
    where $\phi(z) = i \cdot \frac{1 + z}{1 - z}$ maps $\Omega$ to $\mathbb{H} \setminus [0, i/3]$, $T(w) = -iw$ rotates the slit to the positive real axis, $\psi(w) = \sqrt{w}$ removes the branch and straightens the domain, and $\chi$ is a final Möbius transformation mapping the resulting domain back to $\mathbb{D}$. Thus, $f(z)$ is a bijective conformal map from $\Omega$ to $\mathbb{D}$.
    \par Generally, once you have a specific bijective conformal map $f : \Omega \to \mathbb{D}$, every other such map is of the form
    $$ g(z) = M(f(z)), $$
    where $M$ is an automorphism of the unit disk. That is, to find all conformal maps from $\Omega$ to $\mathbb{D}$, find one such map and precompose with all automorphisms of the image domain $\mathbb{D}$.
\end{newproof}

\begin{statement}[4]
    Let $\Omega \neq \mathbb{C}$ be an unbounded region. Is there an analytic isomorphism from $\Omega$ to $\mathbb{C}$? If yes, exhibit one such isomorphism. If no, explain why.
\end{statement}
\begin{newproof}
    No, and suppose for contradiction that there exists a bijective holomorphic map $ f : \Omega \longrightarrow \mathbb{C}. $ Then $f$ would be a nonconstant entire function (because $\Omega$ is open and connected, and if $f$ is holomorphic and surjective onto $\mathbb{C}$, it must be entire). However, $f$ must be injective, and we have previously shown that any entire injective function must be linear. But an affine linear function is surjective on all of $\mathbb{C}$, so $f$ must have domain $\mathbb{C}$ — not just $\Omega \neq \mathbb{C}$. Since $\Omega \subsetneq \mathbb{C}$, $f$ cannot extend to an entire function on all of $\mathbb{C}$. Therefore, there cannot exist a bijective conformal map from $\Omega$ to $\mathbb{C}$ if $\Omega \neq \mathbb{C}$.
\end{newproof}

\begin{statement}[5]
    Let $\Omega = \{ z = x + iy : 0 < x < 1,\ y \in \mathbb{R} \}$. Is there an analytic isomorphism from $\Omega$ to $\mathbb{C}$? If yes, exhibit one such isomorphism. If no, explain why.
\end{statement}
\begin{newproof}
    Yes. Consider
    $$ g(z) = \frac{1}{\exp(2\pi i z)}, $$
    which maps $\Omega$ onto $\mathbb{C} \setminus \{0\}$. Now, since $\mathbb{C} \setminus \{0\}$ is simply connected minus a point, and $\Omega$ is simply connected, the Riemann mapping theorem implies that such an analytic isomorphism exists.
\end{newproof}

\begin{statement}[6]
    Let $\Omega = \mathbb{C} \setminus [0, \infty)$. Is there an analytic isomorphism from $\Omega$ to $\mathbb{C}$? If yes, exhibit one such isomorphism. If no, explain why.
\end{statement}
\begin{newproof}
    Yes. Define
    $$ f(z) = \sqrt{z}, $$
    where $\sqrt{z}$ is the principal branch of the square root: the branch cut is taken along $[0,\infty)$ so that $\sqrt{z}$ is holomorphic on $\Omega$. Then $f$ is holomorphic on $\Omega$, $f$ is injective on $\Omega$, and $f$ maps $\Omega$ onto $\mathbb{C} \setminus (-\infty, 0]$, a slit plane. Then, applying $\log$ or another conformal map, we can move from the slit plane to $\mathbb{C}$.
\end{newproof}

\end{document}