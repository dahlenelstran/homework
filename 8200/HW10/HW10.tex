\documentclass[12pt]{article}
 
\usepackage[margin=.5in]{geometry} 
\usepackage{amsmath,amsthm,amssymb,outlines}
\usepackage{graphicx,tikzsymbols,tcolorbox}
\renewcommand\qedsymbol{$\blacksquare$}
\newcommand{\Z}{\mathbb{Z}}
\newcommand{\R}{\mathbb{R}}
\tcbuselibrary{most}

\newtcolorbox{newtitle}{
  enhanced,
  colframe=black,
  colback=white,
  boxsep=5pt,
  arc=8pt,
  sharp corners=south,
  borderline={0.5pt}{0pt}{black},
  borderline={1.8pt}{-5pt}{black},
  after skip=30pt
}

\newtcolorbox[auto counter]{statement}[1][]{
  enhanced,
  breakable,
  title={Exercise \ifx\\#1\\\thetcbcounter\else#1\fi},
  colframe=black,
  colback=white,
  colbacktitle=white,
  fonttitle=\bfseries,
  coltitle=black,
  attach boxed title to top left={yshift=-0.25mm-\tcboxedtitleheight/2,yshifttext=2mm-\tcboxedtitleheight/2, xshift=2mm},
  boxed title style={boxrule=0.5mm}
}

\newtcolorbox{newproof}{
  enhanced,
  breakable,
  frame hidden,
  colback=white,
  title={Proof.},
  fonttitle=\bfseries,
  coltitle=black,
  colbacktitle=white,
  boxed title style={boxrule=0.5mm},
  attach boxed title to top left={yshift=-0.25mm-\tcboxedtitleheight/2,yshifttext=2mm-\tcboxedtitleheight/2, xshift=2mm},
  borderline west={1.5pt}{8pt}{black},
  after upper={\hfill $\blacksquare$}
}

\begin{document}

\begin{newtitle}
  \begin{center}
    \textbf{\Huge Homework 10}
  \end{center}
  \textbf{} \hfill \textbf{\today}
\end{newtitle}

\begin{statement}[2.2.10]
  Let $X$ be the quotient space of $S^2$ under the identifications $x \sim -x$ for $x$ in the equator $S^1$. Compute 
  the homology groups $H_i(X)$. Do the same for $S^3$ with antipodal points of the equatorial $S^2 \subset S^3$ 
  identified. 
\end{statement}
\begin{newproof}
  For the first $X$, because we have 1 0-cell, 1 1-cell, and 2 2-cells, we get the following chain complex: 
  $$ 0 \xrightarrow{f} \Z \oplus \Z \xrightarrow{g} \Z \xrightarrow{h} \Z \xrightarrow{\phi} 0 $$
  So that our homology groups are:
  \begin{align*}
    H_0(X) \cong \text{ker}(\phi) \text{ / im}(h) \\
    H_1(X) \cong \text{ker}(h) \text{ / im}(g) \\
    H_2(X) \cong \text{ker}(g) \text{ / im}(f) \\
  \end{align*}
  with all other homology groups for $n > 2$ zero.
  \par Clearly, $\text{ker}(\phi) \cong \Z$, and because $h$ is a map from 1-cells to 0-cells, $h$ is the zero map. This is also 
  evident because the boundary of the 1-cell is zero, because the ends are identified. Thus $\text{im}(h) \cong 0$, and 
  $\text{ker}(h) \cong \Z$. 
  \par For the $g$ map, it's easiest to first think about what the map is doing. Because the equator has $x \sim-x$, 
  the 2-cell "wraps around" the equator twice, for both the north and south 2-cells. Thus both generators of $\Z \oplus \Z$, $(1,0)$ 
  and $(0,1)$, map to $2$. Thus the image of $g$ is the map generated by 2, so $\text{im}(g) \cong 2\Z$. When considering the kernel, 
  the elements are of the form $a(1,0)+b(0,1) = 0$, so that $b=-a$. Thus the kernel is wholly generated by $(-1,1)$, so 
  $\text{ker}(g) \cong \Z$. Finally, we can clearly see that $\text{im}(f) \cong 0$. 
  \par Thus we are left with:
  \begin{align*}
    H_0(X) \cong \text{ker}(\phi) \text{ / im}(h) \cong \Z / 0 \cong \Z \\
    H_1(X) \cong \text{ker}(h) \text{ / im}(g) \cong \Z / \ 2\Z \\
    H_2(X) \cong \text{ker}(g) \text{ / im}(f) \cong \Z / 0 \cong \Z\\
  \end{align*}
  Now letting $X$ be the second space, we have a similar CW Complex structure: 1 0-cell, 1 1-cell, 1 2-cell, 2 3-cells. Thus we have the 
  following chain complex:
  $$ 0 \xrightarrow{f} \Z \oplus \Z \xrightarrow{g} \Z \xrightarrow{h} \Z \xrightarrow{\phi} \Z \xrightarrow{\psi} 0 $$
  So that our homology groups are:
  \begin{align*}
    H_0(X) \cong \text{ker}(\psi) \text{ / im}(\phi) \\
    H_1(X) \cong \text{ker}(\phi) \text{ / im}(h) \\
    H_2(X) \cong \text{ker}(h) \text{ / im}(g) \\
    H_3(X) \cong \text{ker}(g) \text{ / im}(f) \\
  \end{align*}
  with all other homology groups for $n > 2$ zero.
  \par We can use a similar logic as the last time to find everything but the map $h$ (note that although we have increased 
  dimension, the 3-cell still "wraps around" the $S^2$ equator twice, similar to before). So we have 
  \begin{align*}
    H_0(X) \cong \text{ker}(\psi) \text{ / im}(\phi) \cong \Z / 0 \cong \Z \\
    H_1(X) \cong \text{ker}(\phi) \text{ / im}(h) \cong \Z /\text{im}(h) \\
    H_2(X) \cong \text{ker}(h) \text{ / im}(g) \cong \text{ker}(h) / 2\Z \\
    H_3(X) \cong \text{ker}(g) \text{ / im}(f) \cong \Z / 0 \cong \Z \\
  \end{align*}
  For the map $h$, as it maps the 2-cell to the 1-cell, 
\end{newproof}

\end{document}
