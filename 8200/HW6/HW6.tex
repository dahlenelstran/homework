\documentclass[12pt]{article}
 
\usepackage[margin=1in]{geometry} 
\usepackage{amsmath,amsthm,amssymb,outlines}
\usepackage{graphicx,tikzsymbols,tcolorbox}
\renewcommand\qedsymbol{$\blacksquare$}
\tcbuselibrary{most}

\newtcolorbox{newtitle}{
  enhanced,
  colframe=black,
  colback=white,
  boxsep=5pt,
  arc=8pt,
  sharp corners=south,
  borderline={0.5pt}{0pt}{black},
  borderline={1.8pt}{-5pt}{black},
  after skip=30pt
}

\newtcolorbox[auto counter]{statement}{
  enhanced,
  title={Exercise \thetcbcounter},
  colframe=black,
  colback=white,
  colbacktitle=white,
  fonttitle=\bfseries,
  coltitle=black,
  attach boxed title to top left={yshift=-0.25mm-\tcboxedtitleheight/2,yshifttext=2mm-\tcboxedtitleheight/2, xshift=2mm},
  boxed title style={boxrule=0.5mm}
}

\newtcolorbox{newproof}{
  enhanced,
  frame hidden,
  colback=white,
  title={Proof.},
  fonttitle=\bfseries,
  coltitle=black,
  colbacktitle=white,
  boxed title style={boxrule=0.5mm},
  attach boxed title to top left={yshift=-0.25mm-\tcboxedtitleheight/2,yshifttext=2mm-\tcboxedtitleheight/2, xshift=2mm},
  borderline west={1.5pt}{8pt}{black},
  after upper={\hfill $\blacksquare$}
}

\begin{document}

\begin{newtitle}
  \begin{center}
    \textbf{\Huge 8200 Homework 6}
  \end{center}
  \hfill \textbf{\today}
\end{newtitle}

\begin{statement}
  Suppose $X$ is path connected and $p: (\tilde{X},\tilde{x_0}) \to (X,x_0)$ is a path connected covering space of $X$. 
  Prove that the number of sheets of this covering space is equal to the index of $p_*(\pi_1(\tilde{X},\tilde{x_0}))$ 
  in $\pi_1(X,x_0)$.
\end{statement}

\begin{newproof}
  Let $f$ be any loop with basepoint $x_0$, so that $\tilde{f}$ is it's lift, where $X$ cooresponds to 
  $\tilde{X}$ and $x_0$, $\tilde{x_0}$. Let $g \in G=p_*(\pi_1(\tilde{X},\tilde{x_0}))$, so that $g \circ f$ has 
  the lift $\tilde{g} \circ \tilde{f}$. Note that because $\tilde{g}$ is a loop, $\tilde{g} \circ 
  \tilde{f}$ ends at the same point as $\tilde{f}$. Then define a function $\phi: G[f] \to p^{-1}(x)$, 
  where $G[f] \mapsto \tilde{f}(1)$. Because $\tilde{X}$ is path connected, $\phi$ is surjective. 
  Then note that because $\phi(G[f_1])=\phi(G[f_2])$ implies that $f_1 \circ \bar{f_2}$ 
  lifts to a loop based at $\tilde{x_0}$, so that $[f_1][f_2]^{-1} \in G$, and $g[f_1]=G[f_2]$, so 
  that $\phi$ is injective. Thus the number of cosets (index) is equal to the number of sheets. 
\end{newproof}

\begin{statement}
  Construct nonnormal covering spaces of the Klein Bottle by a Klein bottle and by a torus. 
\end{statement}

\begin{newproof}

\end{newproof}

\begin{statement}
  Let $X$ be the space obtained from a torus $S^1 \times S^1$ by attaching a Mobius band via a 
  homeomorphism from the boundary circle of the Mobius band to the circle $S^1 \times \{x_0\}$ 
  in the torus. Compute $\pi_1(X)$, describe the universal cover of $X$, and describe the action 
  of $\pi_1(X)$ on the universal cover. Do the same for the space $Y$ obtained by attaching 
  a Mobius band to $\mathbb{R}P^2$ formed by the 1-skeleton of the usual CW structure on $\mathbb{R}P^2$. 
\end{statement}

\begin{newproof}

\end{newproof}

\begin{statement}
  Let $\phi: \mathbb{R}^2 \to \mathbb{R}^2$ be the linear transformation $\phi(x,y)=(2x,y/2)$. This generates 
  an action of $\mathbb{Z}$ on $X = \mathbb{R} - \{0\}$. Show this action is a covering space action and 
  compute $\pi_1(X \setminus \mathbb{Z})$. Show the orbit space $X \setminus \mathbb{Z}$ is 
  non-Hausdorff, and describe how it is a union of four subspaces homeomorphic to $S^1 \times \mathbb{R}$, coming 
  from the complementary components of the $x$-axis and the $y$-axis. 
\end{statement}

\begin{newproof}

\end{newproof}

\begin{statement}
   For a covering space $p: \tilde{X} \to X$ connected, locally path-connected, and semilocally simply-connected, 
   show:
   \begin{itemize}
     \item The components of $\tilde{X}$ are in one-to-one corrspondence with the orbits of the action of 
\end{statement}

\begin{newproof}

\end{newproof}

\begin{statement}

\end{statement}

\begin{newproof}

\end{newproof}

\end{document}
