\documentclass[12pt]{article}
 
\usepackage[margin=1in]{geometry} 
\usepackage{amsmath,amsthm,amssymb,outlines}
\usepackage{graphicx,tikzsymbols,tcolorbox}
\renewcommand\qedsymbol{$\blacksquare$}
\tcbuselibrary{most}

\newtcolorbox{newtitle}{
  enhanced,
  colframe=black,
  colback=white,
  boxsep=5pt,
  arc=8pt,
  sharp corners=south,
  borderline={0.5pt}{0pt}{black},
  borderline={1.8pt}{-5pt}{black},
  after skip=30pt
}

\newtcolorbox[auto counter]{statement}{
  enhanced,
  title={Exercise \thetcbcounter},
  colframe=black,
  colback=white,
  colbacktitle=white,
  fonttitle=\bfseries,
  coltitle=black,
  attach boxed title to top left={yshift=-0.25mm-\tcboxedtitleheight/2,yshifttext=2mm-\tcboxedtitleheight/2, xshift=2mm},
  boxed title style={boxrule=0.5mm}
}

\newtcolorbox{newproof}{
  enhanced,
  frame hidden,
  colback=white,
  title={Proof.},
  fonttitle=\bfseries,
  coltitle=black,
  colbacktitle=white,
  boxed title style={boxrule=0.5mm},
  attach boxed title to top left={yshift=-0.25mm-\tcboxedtitleheight/2,yshifttext=2mm-\tcboxedtitleheight/2, xshift=2mm},
  borderline west={1.5pt}{8pt}{black},
  after upper={\hfill $\blacksquare$}
}

\begin{document}

\begin{newtitle}
  \begin{center}
    \textbf{\Huge 8200 Homework 6}
  \end{center}
  \hfill \textbf{\today}
\end{newtitle}

\begin{statement}
  Suppose $X$ is path connected and $p: (\tilde{X},\tilde{x_0}) \to (X,x_0)$ is a path connected covering space of $X$. 
  Prove that the number of sheets of this covering space is equal to the index of $p_*(\pi_1(\tilde{X},\tilde{x_0}))$ 
  in $\pi_1(X,x_0)$.
\end{statement}

\begin{newproof}
  Let $f$ be any loop with basepoint $x_0$, so that $\tilde{f}$ is it's lift, where $X$ cooresponds to 
  $\tilde{X}$ and $x_0$, $\tilde{x_0}$. Let $g \in G=p_*(\pi_1(\tilde{X},\tilde{x_0}))$, so that $g \circ f$ has 
  the lift $\tilde{g} \circ \tilde{f}$. Note that because $\tilde{g}$ is a loop, $\tilde{g} \circ 
  \tilde{f}$ ends at the same point as $\tilde{f}$. Then define a function $\phi: G[f] \to p^{-1}(x)$, 
  where $G[f] \mapsto \tilde{f}(1)$. Because $\tilde{X}$ is path connected, $\phi$ is surjective. 
  Then note that because $\phi(G[f_1])=\phi(G[f_2])$ implies that $f_1 \circ \bar{f_2}$ 
  lifts to a loop based at $\tilde{x_0}$, so that $[f_1][f_2]^{-1} \in G$, and $g[f_1]=G[f_2]$, so 
  that $\phi$ is injective. Thus the number of cosets (index) is equal to the number of sheets. 
\end{newproof}

\begin{statement}
  Construct nonnormal covering spaces of the Klein Bottle by a Klein bottle and by a torus. 
\end{statement}

\begin{newproof}
  For the Klein bottle, we need to find a nonnormal subgroup of the Klein bottle. The Klein 
  bottle's fundamental group is $\langle a,b \vert aba^{-1}b \rangle$. A nonnormal subgroup 
  of this could be $\langle a,b^2 \rangle$, thus there is a nonnormal covering space that 
  cooresponds to it. We know this covering space is a Klein bottle because the subgroup 
  $\langle a, b^2 \rangle$ is isomorphic to the Klein bottle's fundamental group.  
  \par For the torus, it's a little bit trickier because the nonnormal subgroup must be 
  isomorphic to the torus's fundamental group, $\langle a,b \vert ab=ba \rangle$. If 
  we choose the subgroup $\langle a^2,b^2 \rangle$, it's isomorphic to the torus 
  fundamental group, and nonnormal in the Klein bottle fundamental group, so we get a 
  corresponding nonnormal cover. 
\end{newproof}

\begin{statement}
  Let $X$ be the space obtained from a torus $S^1 \times S^1$ by attaching a Mobius band via a 
  homeomorphism from the boundary circle of the Mobius band to the circle $S^1 \times \{x_0\}$ 
  in the torus. Compute $\pi_1(X)$, describe the universal cover of $X$, and describe the action 
  of $\pi_1(X)$ on the universal cover. Do the same for the space $Y$ obtained by attaching 
  a Mobius band to $\mathbb{R}P^2$ formed by the 1-skeleton of the usual CW structure on $\mathbb{R}P^2$. 
\end{statement}

\begin{newproof}
  Pressed for time and saved this one till the end, so I will just say you have to use Van Kampen 
  to find the fundamental group of this shape, which is just the mobius band wrapping around the 
  torus twice. If I had to guess, the fundamental group would be $\mathbb{Z} * \mathbb{Z}$ or 
  something along those lines.
\end{newproof}

\begin{statement}
  Let $\phi: \mathbb{R}^2 \to \mathbb{R}^2$ be the linear transformation $\phi(x,y)=(2x,y/2)$. This generates 
  an action of $\mathbb{Z}$ on $X = \mathbb{R} - \{0\}$. Show this action is a covering space action and 
  compute $\pi_1(X \setminus \mathbb{Z})$. Show the orbit space $X \setminus \mathbb{Z}$ is 
  non-Hausdorff, and describe how it is a union of four subspaces homeomorphic to $S^1 \times \mathbb{R}$, coming 
  from the complementary components of the $x$-axis and the $y$-axis. 
\end{statement}

\begin{newproof}
  The action of $\mathbb{Z}$ on $X$ is given by:
$$ n \cdot (x, y) = \phi^n(x, y) = (2^n x, 2^{-n} y). $$ 
\par For each $(x, y) \in X$, choose a neighborhood $U$ small enough so that 
$\phi^n(U) \cap U = \emptyset$ for all $n \neq 0$. This is possible because $\phi^n$ scales $x$ 
by $2^n$ and $y$ by $2^{-n}$, ensuring disjointness for small $U$. Thus, the action is a covering space action.
\par Since the action is free and properly discontinuous, the quotient map $X \to X / \mathbb{Z}$ is a covering map 
with deck transformation group $\mathbb{Z}$. Therefore:
$$ \pi_1(X / \mathbb{Z}) \cong \mathbb{Z}. $$
\par Consider the orbits of $(1, 0)$ and $(0, 1)$. The orbit of $(1, 0)$ is $\{(2^n, 0) \mid n \in
\mathbb{Z}\}$, and the orbit of $(0, 1)$ is $\{(0, 2^{-n}) \mid n \in \mathbb{Z}\}$. These orbits accumulate
at $(0, 0)$, which is not in $X$, so their images in $X / \mathbb{Z}$ cannot be separated by disjoint open
sets. Thus, $X / \mathbb{Z}$ is non-Hausdorff.
\par The space $X$ decomposes into four quadrants based on the $x$-axis and $y$-axis. Each quadrant 
is homeomorphic to $S^1 \times \mathbb{R}$, as the angular component corresponds to $S^1$ and 
the radial component to $\mathbb{R}$. The action preserves these quadrants, so $X / \mathbb{Z}$ is a 
union of four subspaces homeomorphic to $S^1 \times \mathbb{R}$.
\end{newproof}

\begin{statement}
  For a covering space $p: \tilde{X} \to X$ connected, locally path-connected, and semilocally simply-connected, 
  show:
  \begin{itemize}
    \item[(a)] The components of $\tilde{X}$ are in one-to-one corrspondence with the orbits of the action of 
      $\pi_1(X,x_o)$ on the fiber $p^{-1}(x_0)$.
    \item[(b)] Under the Galois corrspondence between connected covering spaces of $X$ and subgroups 
      of $\pi_1(X,x_0)$, the subgroup corresponding to the component of $\tilde{X}$ containing
      a given lift $\tilde{x_0}$ of $x_0$ is the \textit{stabilizer} of $\tilde{x_0}$, the 
      subgroup consisting of elements whose action on the fiber leaves $\tilde{x_0}$ fixed.
  \end{itemize}
\end{statement}

\begin{newproof}
  \begin{itemize}
    \item[(a)] Let $p: \tilde{X} \to X$ be a connected, locally path-connected, and semilocally simply-connected 
      covering space. Fix a basepoint $x_0 \in X$ and consider the fiber $p^{-1}(x_0)$. The fundamental
      group $\pi_1(X, x_0)$ has the following action  on $p^{-1}(x_0)$: for $[\gamma] \in 
      \pi_1(X, x_0)$ and $\tilde{x}_0 
      \in p^{-1}(x_0)$, the action is defined by lifting $\gamma$ to a path in $\tilde{X}$ starting at 
      $\tilde{x}_0$ and taking its endpoint. Each component of $\tilde{X}$ is path-connected, and the 
      restriction of $p$ to a component is a covering map. For a fixed $\tilde{x}_0 \in p^{-1}(x_0)$, the 
      orbit of $\tilde{x}_0$ under the action of $\pi_1(X, x_0)$ consists of all points in $p^{-1}(x_0)$ 
      that lie in the same component of $\tilde{X}$ as $\tilde{x}_0$. This is because we have two cases:
      \begin{itemize}
        \item If $\tilde{x}_1$ is in the same component as $\tilde{x}_0$, there is a path $\tilde{\gamma}$ in $\tilde{X}$ from $\tilde{x}_0$ to $\tilde{x}_1$. Projecting $\tilde{\gamma}$ to $X$ gives a loop $\gamma$ in $X$ based at $x_0$, and the action of $[\gamma]$ on $\tilde{x}_0$ sends it to $\tilde{x}_1$.
        \item Conversely, if $\tilde{x}_1$ is in the orbit of $\tilde{x}_0$, there exists a loop $\gamma$ in $X$ such that the lift of $\gamma$ starting at $\tilde{x}_0$ ends at $\tilde{x}_1$. This implies $\tilde{x}_0$ and $\tilde{x}_1$ are in the same component of $\tilde{X}$.
      \end{itemize}
      Thus, the components of $\tilde{X}$ are in one-to-one correspondence with the orbits of the action 
      of $\pi_1(X, x_0)$ on $p^{-1}(x_0)$.
    \item[(b)] Under the Galois correspondence, connected covering spaces of $X$ correspond to subgroups of
      $\pi_1(X, x_0)$. Let $\tilde{X}_0$ be the component of $\tilde{X}$ containing a given lift 
      $\tilde{x}_0$ of $x_0$. The subgroup of $\pi_1(X, x_0)$ corresponding to $\tilde{X}_0$ is the 
      stabilizer of $\tilde{x}_0$.
      The stabilizer of $\tilde{x}_0$ is the subgroup $H \leq \pi_1(X, x_0)$ consisting of elements 
      $[\gamma]$ such that the lift of $\gamma$ starting at $\tilde{x}_0$ ends at $\tilde{x}_0$. Thus it 
      is sufficient to show that this subgroup $H$ cooresponds to the covering space $\tilde{X}_0$.
      \par The covering map $p|_{\tilde{X}_0}: \tilde{X}_0 \to X$ has $H$ as its fundamental group. This follows 
      from the lifting criterion: loops in $X$ lift to loops in $\tilde{X}_0$ if and only if they are in $H$. 
      By Galois correspondence, $H$ is the subgroup associated with $\tilde{X}_0$.
      Thus, the subgroup corresponding to $\tilde{X}_0$ is the stabilizer of $\tilde{x}_0$.
  \end{itemize}
\end{newproof}

\begin{statement} 
  Consider covering spaces $p:\tilde{X} \to X$ with $\tilde{X}$ and $X$ connected CW complexes,
  the cells of $\tilde{X}$ projecting homeomorphically onto cells of $X$. Restricting $p$ to the 
  1-skeleton then gives a covering space $\tilde{X}^1 \to X$ over the 1-skeleton of $X$. Show:
  \begin{itemize}
    \item[(a)] Two such covering spaces $\tilde{X}_1 \to X$ and $\tilde{X}_2 \to X$ are isomorphic 
      if and only if the restrictions $\tilde{X}^1_1 \to X^1$ and $\tilde{X}^1_2 \to X^1$ are 
      isomorphic. 
    \item[(b)] $\tilde{X} \to X$ is a normal covering if and only if $\tilde{X}^1 \to X^1$ is normal.
    \item[(c)] The groups of deck transformations of the coverings $\tilde{X} \to X$ and 
      $\tilde{X}^1 \to X^1$ are isomorphic, via the restriction map. 
  \end{itemize}
\end{statement}

\begin{newproof}
  \begin{itemize}
    \item[(a)] Let $p_1: \tilde{X}_1 \to X$ and $p_2: \tilde{X}_2 \to X$ be covering spaces with $\tilde{X}_1$ and $\tilde{X}_2$ 
      connected CW complexes, and assume the cells of $\tilde{X}_1$ and $\tilde{X}_2$ project 
      homeomorphically onto cells of $X$. Let $\tilde{X}^1_1 \to X^1$ and $\tilde{X}^1_2 \to X^1$ be the 
      restrictions to the 1-skeletons.
      \par ($\Rightarrow$) If $\tilde{X}_1 \to X$ and $\tilde{X}_2 \to X$ are isomorphic, there exists a 
      homeomorphism $f: \tilde{X}_1 \to \tilde{X}_2$ such that $p_2 \circ f = p_1$. Restricting $f$ to the 
      1-skeletons gives a homeomorphism $f|_{\tilde{X}^1_1}: \tilde{X}^1_1 \to \tilde{X}^1_2$ that commutes 
      with the covering maps, so $\tilde{X}^1_1 \to X^1$ and $\tilde{X}^1_2 \to X^1$ are isomorphic.
      \par ($\Leftarrow$) If $\tilde{X}^1_1 \to X^1$ and $\tilde{X}^1_2 \to X^1$ are isomorphic, let $g: 
      \tilde{X}^1_1 \to \tilde{X}^1_2$ be a homeomorphism such that $p_2 \circ g = p_1|_{\tilde{X}^1_1}$. 
      Since $\tilde{X}_1$ and $\tilde{X}_2$ are CW complexes and the cells project homeomorphically, $g$ 
      extends uniquely to a homeomorphism $f: \tilde{X}_1 \to \tilde{X}_2$ satisfying $p_2 \circ f = p_1$. 
      Thus, $\tilde{X}_1 \to X$ and $\tilde{X}_2 \to X$ are isomorphic.
    \item[(b)] ($\Rightarrow$) If $\tilde{X} \to X$ is normal, then for any two lifts $\tilde{x}_1, \tilde{x}_2 \in p^{-1}(x)$ of a point 
      $x \in X$, there exists a deck transformation $f: \tilde{X} \to \tilde{X}$ such that $f(\tilde{x}_1) = 
      \tilde{x}_2$. Restricting $f$ to $\tilde{X}^1$ gives a deck transformation of $\tilde{X}^1 \to X^1$, so 
      $\tilde{X}^1 \to X^1$ is normal.
      \par ($\Leftarrow$) If $\tilde{X}^1 \to X^1$ is normal, then for any two lifts $\tilde{x}_1, \tilde{x}_2 
      \in p^{-1}(x)$ of a point $x \in X^1$, there exists a deck transformation $g: \tilde{X}^1 \to 
      \tilde{X}^1$ such that $g(\tilde{x}_1) = \tilde{x}_2$. Since $\tilde{X}$ is a CW complex and the cells 
      project homeomorphically, $g$ extends uniquely to a deck transformation $f: \tilde{X} \to \tilde{X}$. 
      Thus, $\tilde{X} \to X$ is normal.
    \item[(c)] Let $\text{Deck}(\tilde{X} \to X)$ and $\text{Deck}(\tilde{X}^1 \to X^1)$ denote the groups of deck transformations of 
      $\tilde{X} \to X$ and $\tilde{X}^1 \to X^1$, respectively. Define the restriction map:
      $$ \Phi: \text{Deck}(\tilde{X} \to X) \to \text{Deck}(\tilde{X}^1 \to X^1), \quad f \mapsto f|_{\tilde{X}^1}. $$ 
      Then note that:
      \begin{itemize}
        \item $\Phi$ is injective: If $f|_{\tilde{X}^1} = g|_{\tilde{X}^1}$, then $f = g$ because $\tilde{X}$ is a CW complex 
          and $f$ and $g$ agree on the 1-skeleton.
        \item $\Phi$ is surjective: For any deck transformation $g: \tilde{X}^1 \to \tilde{X}^1$, $g$ extends uniquely to a 
          deck transformation $f: \tilde{X} \to \tilde{X}$ because the cells of $\tilde{X}$ project 
          homeomorphically onto cells of $X$.
      \end{itemize}
      Thus, $\Phi$ is an isomorphism, and the groups of deck transformations are isomorphic.
  \end{itemize}
\end{newproof}

\end{document}
