\documentclass[12pt]{article} 
\usepackage[margin=1in]{geometry} 
\usepackage{amsmath,amsthm,amssymb,outlines}
\usepackage{graphicx}
\usepackage{tikzsymbols}
\newtheorem{definition}{Definition}
\newtheorem{question}{Question}
\newtheorem{answer}{Answer}
\newtheorem{example}{Example}
\newtheorem{theorem}{Theorem}
\newenvironment{statement}[2][Statement]{\begin{trivlist}
\item[\hskip \labelsep {\bfseries #1}\hskip \labelsep {\bfseries #2.}]}{\end{trivlist}}

\begin{document}
 
\title{Topology Notes} 
\author{Dahlen Elstran} 
\maketitle

\section*{Chapter 1}

\subsection*{1.3: Covering Spaces}

\begin{definition}
  A \textbf{covering space} of a space $X$ is a space $\tilde{X}$ together with a map $p: \tilde{X} \to X$ satisfying the following condition: 
  Each point $x \in X$ has an open neighborhood $U$ in $X$ such that $p^{-1}(U)$ is a union of disjoint open sets in $\tilde{X}$, 
  each of which is mapped homeomorphically onto $U$ by $p$.
\end{definition}

So a covering space is just a space such that every open neighborhood of any point in $X$ has an open neighborhood covering it in $\tilde{X}$. 
Something being mapped homeomorphically just means that it retains the shape in a topological sense. ChatGPT explained it as a set of open sets in $X$. 
These open sets completely cover $X$, with nothing more. So the union of all the open sets in the open cover is exactly $X$. 

\begin{definition}
  In the previous definition, the space $U$ is called \textbf{evenly covered}.
\end{definition}

\begin{definition}
  The dijoint open sets in $\tilde{X}$ that project homeomorphically to $U$ by $p$ are called \textbf{sheets} of $\tilde{X}$ over $U$.
\end{definition}

So a sheet is just $U$'s matching open set in the cover. The number of sheets a cover has corresponds to the number of times something in $X$ is covered. 

\begin{definition}
  A \textbf{lift} of a map $f: Y \to X$ is a map $\tilde{f}:Y \to \tilde{X}$ such that $p \tilde{f} = f$. 
\end{definition}

If $p: \tilde{X} \to X$, then $p \tilde{f} : Y \to \tilde{X} \to X$. So pretty much what's happening is a lift is a way to relate a map 
from one space to another to one space and the other space's cover. 

\begin{theorem}[Homotopy Lifting Property/ Covering Homotopy Property]
  Given a covering space $p: \tilde{X} \to X$, a homotopy $f_t:Y \to X$, and a map $\tilde{f_0}:Y \to \tilde{X}$ lifting $f_0$, then 
  there exists a uniqye homotopy $\tilde{f_t}: Y \to \tilde{X}$ of $f_0$ that lifts $f_t$. 
\end{theorem}

All this is saying is that if there exists two spaces with a homotopy between them, and the first space can be lifted to the second, then there is a homotopy between the first 
space and the cover of the second. 

\par If you let $Y$ be a point, you get the  \textbf{path lifting property}.

\par Also note that the uniqueness of lifts implies that every lift of a constant path is constant.

\begin{theorem}
  The map $p_*:\pi_1(\tilde{X},\tilde{x_0}) \to \pi_1(X, x_0)$ induced by a covering space $p:(\tilde{X},\tilde{x_0}) \to (X,x_0)$ is injective. 
  The image subgroup $p_*(\pi_1(\tilde{X}, \tilde{x_0})$ in $\pi_1(X,x_0)$ consists of the homotopy classes of loops in $X$ based at $x_0$ whose lifts 
  to $\tilde{X}$ starting at $\tilde{x_0}$ are loops.
\end{theorem}

So because lifts are unique, the induced homomorphisms from them must be injective, makes sense.

\begin{theorem}
  The number of sheets of a covering space $p:(\tilde{X}, \tilde{x_0}) \to (X,x_0)$ with $X$ and $\tilde{X}$ path-connected equals the index of $p_*(\pi_1(\tilde{X},\tilde{x_0}))$ in $\pi_1(X,x_0)$. 
\end{theorem}

The number of sheets is just how many times things in $X$ are covered, and an index is how many left (or right) cosets of a subgroup fit in a group. So if a loop in $X$ has 4 sheets, or is 
covered 4 times, naturally, there are 4 cosets of $p_*(\pi_1(\tilde{X},\tilde{x_0}))$ in $\pi_1(X,x_0)$.

\begin{theorem}[Lifting Criterion]
  Suppose given a covering space $p: (\tilde{X}, \tilde{x_0}) \to (X,x_0)$ and a map $f:(Y,y_0) \to (X,x_0)$ with $Y$ path-connected and locally path-connected. Then a lift
  $\tilde{f}:(Y,y_0) \to (\tilde{X}, \tilde{x_0})$ of $f$ exists if and only if $f_*(\pi_1(Y,y_0)) \subset p_*(\pi_1(\tilde{X}, \tilde{x_0}))$.
\end{theorem}

\begin{theorem}[Unique Lifting Property]
  Given a covering space $p: \tilde{X} \to X$ and a map $f: Y \to X$, if two lifts $\tilde{f_1}, \tilde{f_2}: Y \to \tilde{X}$ 
  of $f$ agree at one point of $Y$ and $Y$ is connected, then $\tilde{f_1}$ and $\tilde{f_2}$ agree on all of $Y$. 
\end{theorem}

To explain both of the previous theorems: First, to be locally path-connected means that any point has arbitrarily small open neighborhoods that are 
path-connected. So all this theorem is saying is that because a point in $X$ can have multiple matching points 
in the open cover, because the cover can have multiple sheets, each matching point creates a unique match to
any path started at the point in $X$. 
\par The Lifting Criterion tells us that we can only lift a path if the starting point has a matching point in the open cover. 
This is usually true, but if the open cover isn't homeomorphic in every neighborhood of every point, things may get freaky. 
The Unique Lifting Property tells us that each sheet will have a unique lifted path. 

FINISH LATER!!!!!

\end{document}
